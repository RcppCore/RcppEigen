
\documentclass[shortnames,article]{jss}
%\VignetteIndexEntry{RcppEigen-intro}
%\VignetteKeywords{linear algebra, template programming, C++, R, Rcpp}
%\VignettePackage{RcppEigen}

%\usepackage{booktabs,flafter,bm,amsmath}
\usepackage{booktabs,bm,amsmath}

\newcommand{\R}{\proglang{R}\ } % NB forces a space so not good before % fullstop etc.
\newcommand{\Cpp}{\proglang{C++}\ }

\author{Douglas Bates\\U Wisconsin/Madison \And Dirk Eddelbuettel\\Debian Project \And Romain Fran\c{c}ois\\R Enthusiasts}
\title{An Introduction to \pkg{RcppEigen}}

\Plainauthor{Douglas Bates, Dirk Eddelbuettel, Romain Fran\c{c}ois} 
\Plaintitle{An Introduction to RcppEigen}
\Shorttitle{An Introduction to RcppEigen}

\Abstract{
  \noindent
  \noindent
  The \pkg{RcppEigen} package provides access from \proglang{R}
  \citep{R:Main} to the \pkg{Eigen} \citep*{Eigen:Web} \proglang{C++}
  template library for numerical linear algebra. \pkg{Rcpp}
  \citep{JSS:Rcpp,CRAN:Rcpp} classes and specializations of the
  \proglang{C++} templated functions \code{as} and \code{wrap} from
  \pkg{Rcpp} provide the ``glue'' for passing objects from \proglang{R} to
  \proglang{C++} and back.  
}

\Keywords{Linear algebra, template programming, \proglang{R}, \proglang{C++}, \pkg{Rcpp}} %% at least one keyword must be supplied
\Plainkeywords{Linear algebra, template programmig, R, C++, Rcpp} %% without formatting

\Address{
  Douglas Bates \\
  Department of Statistics \\
  University of Wisconsin - Madison \\
  Madison, WI, USA \\
  E-mail: \email{bates@stat.wisc.edu} \\
  URL: \url{http://www.stat.wisc.edu/~bates/}\\

  Dirk Eddelbuettel \\
  Debian Project \\
  River Forest, IL, USA\\
  E-mail: \email{edd@debian.org}\\
  URL: \url{http://dirk.eddelbuettel.com}\\
  
  Romain Fran\c{c}ois\\
  Professional R Enthusiast\\
  1 rue du Puits du Temple, 34 000 Montpellier\\
  FRANCE \\
  E-mail: \email{romain@r-enthusiasts.com}\\
  URL: \url{http://romainfrancois.blog.free.fr}
}

%% need no \usepackage{Sweave.sty}

\newcommand{\rank}{\operatorname{rank}}

%% highlights macros
%% Style definition file generated by highlight 2.7, http://www.andre-simon.de/
% Highlighting theme definition:
\newcommand{\hlstd}[1]{\textcolor[rgb]{0,0,0}{#1}}
\newcommand{\hlnum}[1]{\textcolor[rgb]{0,0,0}{#1}}
\newcommand{\hlopt}[1]{\textcolor[rgb]{0,0,0}{#1}}
\newcommand{\hlesc}[1]{\textcolor[rgb]{0.74,0.55,0.55}{#1}}
%\newcommand{\hlstr}[1]{\textcolor[rgb]{0.74,0.55,0.55}{#1}}
\newcommand{\hlstr}[1]{\textcolor[rgb]{0.90,0.15,0.15}{#1}}
%green: \newcommand{\hlstr}[1]{\textcolor[rgb]{0.13,0.67,0.13}{#1}} % 0.74 -> % 0.90; 0.55 -> 0.25
\newcommand{\hldstr}[1]{\textcolor[rgb]{0.74,0.55,0.55}{#1}}
\newcommand{\hlslc}[1]{\textcolor[rgb]{0.67,0.13,0.13}{\it{#1}}}
\newcommand{\hlcom}[1]{\textcolor[rgb]{0.67,0.13,0.13}{\it{#1}}}
\newcommand{\hldir}[1]{\textcolor[rgb]{0,0,0}{#1}}
\newcommand{\hlsym}[1]{\textcolor[rgb]{0,0,0}{#1}}
\newcommand{\hlline}[1]{\textcolor[rgb]{0.33,0.33,0.33}{#1}}
\newcommand{\hlkwa}[1]{\textcolor[rgb]{0.61,0.13,0.93}{\bf{#1}}}
\newcommand{\hlkwb}[1]{\textcolor[rgb]{0.13,0.54,0.13}{#1}}
\newcommand{\hlkwc}[1]{\textcolor[rgb]{0,0,1}{#1}}
\newcommand{\hlkwd}[1]{\textcolor[rgb]{0,0,0}{#1}}
\definecolor{bgcolor}{rgb}{1,1,1}


% ------------------------------------------------------------------------

\begin{document}

%\SweaveOpts{engine=R,eps=FALSE}

\section{Introduction}
\label{sec:intro}

Linear algebra is an essential building block of statistical
computing.  Operations such as matrix decompositions, linear program
solvers, and eigenvalue / eigenvector computations are used in many
estimation and analysis routines. As such, libraries supporting linear
algebra have long been provided by statistical programmers for
different programming languages and environments.  Because it is
object-oriented, \proglang{C++}, one of the central modern languages
for numerical and statistical computing, is particularly effective at
representing matrices, vectors and decompositions, and numerous class
libraries providing linear algebra routines have been written over the
years.
% Could cite Eddelbuettel (1996) here, but a real survey would be better.

As both the \proglang{C++} language and standards have evolved
\citep{Meyers:2005:EffectiveC++,Meyers:1995:MoreEffectiveC++}, so have the
compilers implementing the language.  Relatively modern language constructs
such as template meta-programming are particularly useful because they provide
overloading of operations (allowing expressive code in the compiled language
similar to what can be done in scripting languages) and can shift some of the
computational burden from the run-time to the compile-time (though a more
detailed discussions of template programming is however beyond this
paper). \cite{Veldhuizen:1998:Blitz} provided an early and influential
implementation that already demonstrated key features of this approach.  Its
usage was held back at the time by the limited availability of
compilers implementing all necessary features of the \proglang{C++}
language.

This situation has greatly improved over the last decade, and many more such
libraries have been contributed. One such \proglang{C++} library is
\pkg{Eigen} by \citet*{Eigen:Web}. \pkg{Eigen} started as a sub-project to
KDE (a popular Linux desktop environment), initially focussing on fixed-size
matrices to represent projections in a visualization application. \pkg{Eigen}
grew from there and has over the course of about a decade produced three
major releases with ``Eigen3'' being the current version.

\pkg{Eigen} is of interest as the \proglang{R} system for statistical
computation and graphics \citep{R:Main} is itself easily extensible. This is
particular true via the \proglang{C} language that most of \proglang{R}'s
compiled core parts are written in, but also for the \proglang{C++} language
which can interface with \proglang{C}-based systems rather easily. The manual
``Writing R Extensions'' \citep{R:Extensions} is the basic reference for
extending \proglang{R} with either \proglang{C} or \proglang{C++}.

The \pkg{Rcpp} package by \citet{JSS:Rcpp,CRAN:Rcpp} facilitates extending
\proglang{R} with \proglang{C++} code by providing seamless object mapping
between both languages.
%
As stated in the \pkg{Rcpp} \citep{CRAN:Rcpp} vignette, ``Extending \pkg{Rcpp}''
\begin{quote}
  \pkg{Rcpp} facilitates data interchange between \proglang{R} and
  \proglang{C++} through the templated functions \texttt{Rcpp::as} (for
  conversion of objects from \proglang{R} to \proglang{C++}) and
  \texttt{Rcpp::wrap} (for conversion from \proglang{C++} to \proglang{R}).
\end{quote}
The \pkg{RcppEigen} package provides the header files composing the
\pkg{Eigen} \proglang{C++} template library and implementations of
\texttt{Rcpp::as} and \texttt{Rcpp::wrap} for the \proglang{C++}
classes defined in \pkg{Eigen}.

The \pkg{Eigen} classes themselves provide high-performance,
versatile and comprehensive representations of dense and sparse
matrices and vectors, as well as decompositions and other functions
to be applied to these objects.  The next section introduces some
of these classes and shows how to interface to them from \proglang{R}.

\section{Eigen classes}
\label{sec:eclasses}

\pkg{Eigen} \citep*{Eigen:Web} is a \proglang{C++} template
library providing classes for many forms of matrices, vectors, arrays
and decompositions.  These classes are flexible and comprehensive
allowing for both high performance and well structured code
representing high-level operations. \proglang{C++} code based on Eigen
is often more like \proglang{R} code, working on the ``whole object'',
than compiled code in other languages where operations often must be
coded in loops.

As in many \proglang{C++} template libraries using template meta-programming
\citep{Abrahams+Gurtovoy:2004:TemplateMetaprogramming}, the templates
themselves can be very complicated.  However, \pkg{Eigen} provides
\code{typedef}s for common classes that correspond to \proglang{R} matrices and
vectors, as shown in Table~\ref{tab:REigen}, and this paper will use these
\code{typedef}s.
\begin{table}[tb]
  \caption{Correspondence between R matrix and vector types and classes in the \code{Eigen} namespace.}
  \label{tab:REigen}
  \centering
  \begin{tabular}{l l}
    \toprule
    \multicolumn{1}{c}{\proglang{R} object type} & \multicolumn{1}{c}{\pkg{Eigen} class typedef}\\
    \midrule
    numeric matrix                          & \code{MatrixXd}\\
    integer matrix                          & \code{MatrixXi}\\
    complex matrix                          & \code{MatrixXcd}\\
    numeric vector                          & \code{VectorXd}\\
    integer vector                          & \code{VectorXi}\\
    complex vector                          & \code{VectorXcd}\\
    \code{Matrix::dgCMatrix} \phantom{XXX}  & \code{SparseMatrix<double>}\\
    \bottomrule
  \end{tabular}
\end{table}

The \proglang{C++} classes shown in Table~\ref{tab:REigen} are in the
\code{Eigen} namespace, which means that they must be written as
\code{Eigen::MatrixXd}.  However, if one prefaces the use of these class
names with a declaration like

%% Alternatively, use 'highlight --enclose-pre --no-doc --latex --style=emacs --syntax=C++'
%% as the command invoked from C-u M-|
%% For version 3.5 of highlight this should be
%%  highlight --enclose-pre --no-doc --out-format=latex --syntax=C++
%%
%% keep one copy to redo later
%%
%% using Eigen::MatrixXd;
%%
\begin{quote}
  \noindent
  \ttfamily
  \hlstd{}\hlkwa{using\ }\hlstd{Eigen}\hlopt{::}\hlstd{MatrixXd}\hlopt{;}\hlstd{}\hspace*{\fill}\\
  \mbox{}
  \normalfont
  \normalsize
\end{quote}
then one can use these names without the namespace qualifier.

\subsection{Mapped matrices in Eigen}
\label{sec:mapped}

Storage for the contents of matrices from the classes shown in
Table~\ref{tab:REigen} is allocated and controlled by the class
constructors and destructors.  Creating an instance of such a class
from an \proglang{R} object involves copying its contents.  An
alternative is to have the contents of the \proglang{R} matrix or
vector mapped to the contents of the object from the Eigen class.  For
dense matrices one can use the Eigen templated class \code{Map}, and for
sparse matrices one can deploy the Eigen templated class \code{MappedSparseMatrix}.

One must, of course, be careful not to modify the contents of the
\proglang{R} object in the \proglang{C++} code.  A recommended
practice is always to declare mapped objects as {\ttfamily\hlkwb{const}\normalfont}.

\subsection{Arrays in Eigen}
\label{sec:arrays}

For matrix and vector classes \pkg{Eigen} overloads the \texttt{`*'}
operator as matrix multiplication.  Occasionally component-wise
operations instead of matrix operations are desired, for which the
\code{Array} templated classes are used in \pkg{Eigen}.  Switching
back and forth between matrices or vectors and arrays is usually done
via the \code{array()} method for matrix or vector objects and the
\code{matrix()} method for arrays.

\subsection{Structured matrices in Eigen}
\label{sec:structured}

There are \pkg{Eigen} classes for matrices with special structure such
as symmetric matrices, triangular matrices and banded matrices.  For
dense matrices, these special structures are described as ``views'',
meaning that the full dense matrix is stored but only part of the
matrix is used in operations.  For a symmetric matrix one needs to
specify whether the lower triangle or the upper triangle is to be used as
the contents, with the other triangle defined by the implicit symmetry.

\section{Some simple examples}
\label{sec:simple}

\proglang{C++} functions to perform simple operations on matrices or
vectors can follow a pattern of:
\begin{enumerate}
\item Map the \proglang{R} objects passed as arguments into Eigen objects.
\item Create the result.
\item Return \code{Rcpp::wrap} applied to the result.
\end{enumerate}

An idiom for the first step is
% using Eigen::Map;
% using Eigen::MatrixXd;
% using Rcpp::as;
%
% const Map<MatrixXd>  A(as<Map<MatrixXd> >(AA));
%\end{lstlisting}
\begin{quote}
  \noindent
  \ttfamily
  \hlstd{}\hlkwa{using\ }\hlstd{Eigen}\hlsym{::}\hlstd{Map}\hlsym{;}\hspace*{\fill}\\
  \hlstd{}\hlkwa{using\ }\hlstd{Eigen}\hlsym{::}\hlstd{MatrixXd}\hlsym{;}\hspace*{\fill}\\
  \hlstd{}\hlkwa{using\ }\hlstd{Rcpp}\hlsym{::}\hlstd{as}\hlsym{;}\hspace*{\fill}\\
  \hlstd{}\hspace*{\fill}\\
  \hlkwb{const\ }\hlstd{Map}\hlsym{$<$}\hlstd{MatrixXd}\hlsym{$>$}\hlstd{\ \ }\hlsym{}\hlstd{}\hlkwd{A}\hlstd{}\hlsym{(}\hlstd{as}\hlsym{$<$}\hlstd{Map}\hlsym{$<$}\hlstd{MatrixXd}\hlsym{$>$\ $>$(}\hlstd{AA}\hlsym{));}\hlstd{}\hspace*{\fill}\\
  \mbox{}
  \normalfont
\end{quote}
where \code{AA} is the name of the R object (called an \code{SEXP} in
\proglang{C} and \proglang{C++}) passed to the \proglang{C++} function.

An alternative to the \code{using} declarations is to declare a \code{typedef} as in
% typedef Eigen::Map<Eigen::MatrixXi>   MapMati;
% const MapMati          A(Rcpp::as<MapMati>(AA));
\begin{quote}
  \noindent
  \ttfamily
  \hlstd{}\hlkwc{typedef\ }\hlstd{Eigen}\hlopt{::}\hlstd{Map}\hlopt{$<$}\hlstd{Eigen}\hlopt{::}\hlstd{MatrixXi}\hlopt{$>$}\hlstd{\ \ \ }\hlopt{}\hlstd{MapMati}\hlopt{;}\hspace*{\fill}\\
  \hlstd{}\hlkwb{const\ }\hlstd{MapMati}\hlstd{\ \ \ \ \ \ \ \ \ \ }\hlstd{}\hlkwd{A}\hlstd{}\hlopt{(}\hlstd{Rcpp}\hlopt{::}\hlstd{as}\hlopt{$<$}\hlstd{MapMati}\hlopt{$>$(}\hlstd{AA}\hlopt{));}\hlstd{}\hspace*{\fill}\\
  \mbox{}
  \normalfont
  \normalsize
\end{quote}

The \code{cxxfunction} from the \pkg{inline} package \citep*{CRAN:inline} for
\proglang{R} and its \pkg{RcppEigen} plugin provide a convenient method of
developing and debugging the \proglang{C++} code.  For actual production code
one generally incorporates the \proglang{C++} source code files in a package
and includes the line \code{LinkingTo: Rcpp, RcppEigen} in the package's
\code{DESCRIPTION} file.  The \code{RcppEigen.package.skeleton} function
provides a quick way of generating the skeleton of a package using
\pkg{RcppEigen} facilities.

The \code{cxxfunction} with the \code{"Rcpp"} or \code{"RcppEigen"}
plugins has the \code{as} and \code{wrap} functions already defined as
\code{Rcpp::as} and \code{Rcpp::wrap}.  In the examples below
these declarations are omitted.  It is important to remember that they are
needed in actual \proglang{C++} source code for a package.

The first few examples are simply for illustration as the operations
shown could be more effectively performed directly in \proglang{R}.
Finally, the results from \pkg{Eigen} are compared to those from the direct
\proglang{R} results.

\subsection{Transpose of an integer matrix}
\label{sec:transpose}

The next \proglang{R} code snippet creates a simple matrix of integers
\begin{verbatim}
> (A <- matrix(1:6, ncol=2))
     [,1] [,2]
[1,]    1    4
[2,]    2    5
[3,]    3    6
> str(A)
 int [1:3, 1:2] 1 2 3 4 5 6
> 
\end{verbatim}
and, in Figure~\ref{trans}, the \code{transpose()} method for the
\code{Eigen::MatrixXi} class is used to return the transpose of the supplied matrix. The \proglang{R}
matrix in the \code{SEXP} \code{AA} is mapped to an
\code{Eigen::MatrixXi} object then the matrix \code{At} is constructed
from its transpose and returned to \proglang{R}.

\begin{figure}[htb]
  %\begin{quote}
    \noindent
    \ttfamily
    \hlstd{}\hlkwa{using\ }\hlstd{Eigen}\hlsym{::}\hlstd{Map}\hlsym{;}\hspace*{\fill}\\
    \hlstd{}\hlkwa{using\ }\hlstd{Eigen}\hlsym{::}\hlstd{MatrixXi}\hlsym{;}\hspace*{\fill}\\
    \hlstd{}\hlstd{\ \ \ \ \ \ \ \ \ \ \ \ \ \ \ \ \ }\hlstd{}\hlslc{//\ Map\ the\ integer\ matrix\ AA\ from\ R}\hspace*{\fill}\\
    \hlstd{}\hlkwb{const\ }\hlstd{Map}\hlsym{$<$}\hlstd{MatrixXi}\hlsym{$>$}\hlstd{\ \ }\hlsym{}\hlstd{}\hlkwd{A}\hlstd{}\hlsym{(}\hlstd{as}\hlsym{$<$}\hlstd{Map}\hlsym{$<$}\hlstd{MatrixXi}\hlsym{$>$\ $>$(}\hlstd{AA}\hlsym{));}\hspace*{\fill}\\
    \hlstd{}\hlstd{\ \ \ \ \ \ \ \ \ \ \ \ \ \ \ \ \ }\hlstd{}\hlslc{//\ evaluate\ and\ return\ the\ transpose\ of\ A}\hspace*{\fill}\\
    \hlstd{}\hlkwb{const\ }\hlstd{MatrixXi}\hlstd{\ \ \ \ \ \ }\hlstd{}\hlkwd{At}\hlstd{}\hlsym{(}\hlstd{A}\hlsym{.}\hlstd{}\hlkwd{transpose}\hlstd{}\hlsym{());}\hspace*{\fill}\\
    \hlstd{}\hlkwa{return\ }\hlstd{}\hlkwd{wrap}\hlstd{}\hlsym{(}\hlstd{At}\hlsym{);}\hlstd{}\hspace*{\fill}
    \normalfont
  %\end{quote}
  \caption{\textbf{transCpp}: Transpose a matrix of integers}
  \label{trans}
\end{figure}

The next \proglang{R} snippet below compiles and links the \proglang{C++} code
segment. The actual work is done by the function \code{cxxfunction} from the \pkg{inline}
package which compiled, links and loads code written in \proglang{C++} and
supplie as text variable.  This frees the user from having to know about
compiler and linker details and options, which makes ``exploratory
programming'' much easier.  Here the program piece to be compiled is stored
as text in a variable named \code{transCpp}), and \code{cxxfunction} createan
an executable function which is assigned to \code{ftrans}.  This new function
is used on the matrix $A$ used above, and one can check that it works as intended
by comparing the output to an explicit transpose of the matrix argument.
\begin{verbatim}
> ftrans <- cxxfunction(signature(AA="matrix"), transCpp, plugin="RcppEigen")
> (At <- ftrans(A))
     [,1] [,2] [,3]
[1,]    1    2    3
[2,]    4    5    6
> stopifnot(all.equal(At, t(A)))
\end{verbatim}

For numeric or integer matrices the \code{adjoint()} method is
equivalent to the \code{transpose()} method.  For complex matrices, the
adjoint is the conjugate of the transpose.  In keeping with the
conventions in the \pkg{Eigen} documentation, in what follows,
the \code{adjoint()} method is used to create the transpose of numeric or
integer matrices.


\subsection{Products and cross-products}
\label{sec:products}

As mentioned in Sec.~\ref{sec:arrays}, the \code{`*'} operator
performs matrix multiplication on \code{Eigen::Matrix} or
\code{Eigen::Vector} objects. The \proglang{C++} code in
Figure~\ref{prod} produces a list containing both the product and
cross-product of its two arguments.

\begin{figure}[htb]
  %\begin{quote}
  \noindent
  \ttfamily
  \hlstd{}\hlkwc{typedef\ }\hlstd{Eigen}\hlopt{::}\hlstd{Map}\hlopt{$<$}\hlstd{Eigen}\hlopt{::}\hlstd{MatrixXi}\hlopt{$>$}\hlstd{\ \ \ }\hlopt{}\hlstd{MapMati}\hlopt{;}\hspace*{\fill}\\
  \hlstd{}\hlkwb{const\ }\hlstd{MapMati}\hlstd{\ \ \ \ }\hlstd{}\hlkwd{B}\hlstd{}\hlopt{(}\hlstd{as}\hlopt{$<$}\hlstd{MapMati}\hlopt{$>$(}\hlstd{BB}\hlopt{));}\hspace*{\fill}\\
  \hlstd{}\hlkwb{const\ }\hlstd{MapMati}\hlstd{\ \ \ \ }\hlstd{}\hlkwd{C}\hlstd{}\hlopt{(}\hlstd{as}\hlopt{$<$}\hlstd{MapMati}\hlopt{$>$(}\hlstd{CC}\hlopt{));}\hspace*{\fill}\\
  \hlstd{}\hlkwa{return\ }\hlstd{List}\hlopt{::}\hlstd{}\hlkwd{create}\hlstd{}\hlopt{(}\hlstd{\textunderscore }\hlopt{{[}}\hlstd{}\hlstr{"B\ \%{*}\%\ C"}\hlstd{}\hlopt{{]}}\hlstd{\ \ \ \ \ \ \ \ \ }\hlopt{=\ }\hlstd{B\ }\hlopt{{*}\ }\hlstd{C}\hlopt{,}\hspace*{\fill}\\
  \hlstd{}\hlstd{\ \ \ \ \ \ \ \ \ \ \ \ \ \ \ \ \ \ \ \ }\hlstd{\textunderscore }\hlopt{{[}}\hlstd{}\hlstr{"crossprod(B,\ C)"}\hlstd{}\hlopt{{]}\ =\ }\hlstd{B}\hlopt{.}\hlstd{}\hlkwd{adjoint}\hlstd{}\hlopt{()\ {*}\ }\hlstd{C}\hlopt{);}\hlstd{}\hspace*{\fill}\\
  \mbox{}
  \normalfont
  \normalsize
  %\end{quote}
  \caption{\textbf{prodCpp}: Product and cross-product of two matrices}
  \label{prod}
\end{figure}

\begin{verbatim}
> fprod <- cxxfunction(signature(BB = "matrix", CC = "matrix"), 
+                      prodCpp, "RcppEigen")
> B <- matrix(1:4, ncol=2)
> C <- matrix(6:1, nrow=2)
> str(fp <- fprod(B, C))
List of 2
 $ B %*% C        : int [1:2, 1:3] 21 32 13 20 5 8
 $ crossprod(B, C): int [1:2, 1:3] 16 38 10 24 4 10
> stopifnot(all.equal(fp[[1]], B %*% C), all.equal(fp[[2]], crossprod(B, C)))
> 
\end{verbatim}

Notice that the \code{create} method for the \pkg{Rcpp} class
\code{List} implicitly applies \code{Rcpp::wrap} to its arguments.



\subsection{Crossproduct of a single matrix}
\label{sec:crossproduct}

As shown in the last example, the \proglang{R} function
\code{crossprod} calculates the product of the transpose of its first
argument with its second argument.  The single argument form,
\code{crossprod(X)}, evaluates $\bm X^\prime\bm X$.  One could, of
course, calculate this product as
\begin{verbatim}
t(X) %*% X
\end{verbatim}
but \code{crossprod(X)} is roughly twice as fast because the result is
known to be symmetric and only one triangle needs to be calculated.
The function \code{tcrossprod} evaluates \code{crossprod(t(X))}
without actually forming the transpose.

To express these calculations in Eigen, a \code{SelfAdjointView} is created,
which is a dense matrix of which only one triangle is used, the other
triangle being inferred from the symmetry.  (``Self-adjoint'' is
equivalent to symmetric for non-complex matrices.)

The \pkg{Eigen} class name is \code{SelfAdjointView}.  The method for
general matrices that produces such a view is called
\code{selfadjointView}.  Both require specification of either the
\code{Lower} or \code{Upper} triangle.

For triangular matrices the class is \code{TriangularView} and the
method is \code{triangularView}.  The triangle can be specified as
\code{Lower}, \code{UnitLower}, \code{StrictlyLower}, \code{Upper},
\code{UnitUpper} or \code{StrictlyUpper}.

For self-adjoint views the \code{rankUpdate} method adds a scalar multiple
of $\bm A\bm A^\prime$ to the current symmetric matrix.  The scalar
multiple defaults to 1.  The code in Figure~\ref{crossprod} produces

\begin{figure}[htb]
  %\begin{quote}
    \noindent
    \ttfamily
    \hlstd{}\hlkwa{using\ }\hlstd{Eigen}\hlopt{::}\hlstd{Map}\hlopt{;}\hspace*{\fill}\\
    \hlstd{}\hlkwa{using\ }\hlstd{Eigen}\hlopt{::}\hlstd{MatrixXi}\hlopt{;}\hspace*{\fill}\\
    \hlstd{}\hlkwa{using\ }\hlstd{Eigen}\hlopt{::}\hlstd{Lower}\hlopt{;}\hspace*{\fill}\\
    \hlstd{}\hspace*{\fill}\\
    \hlkwb{const\ }\hlstd{Map}\hlopt{$<$}\hlstd{MatrixXi}\hlopt{$>$\ }\hlstd{}\hlkwd{A}\hlstd{}\hlopt{(}\hlstd{as}\hlopt{$<$}\hlstd{Map}\hlopt{$<$}\hlstd{MatrixXi}\hlopt{$>$\ $>$(}\hlstd{AA}\hlopt{));}\hspace*{\fill}\\
    \hlstd{}\hlkwb{const\ int}\hlstd{\ \ \ \ \ \ \ \ \ \ \ }\hlkwb{}\hlstd{}\hlkwd{m}\hlstd{}\hlopt{(}\hlstd{A}\hlopt{.}\hlstd{}\hlkwd{rows}\hlstd{}\hlopt{()),\ }\hlstd{}\hlkwd{n}\hlstd{}\hlopt{(}\hlstd{A}\hlopt{.}\hlstd{}\hlkwd{cols}\hlstd{}\hlopt{());}\hspace*{\fill}\\
    \hlstd{MatrixXi}\hlstd{\ \ \ \ \ \ \ \ \ \ }\hlstd{}\hlkwd{AtA}\hlstd{}\hlopt{(}\hlstd{}\hlkwd{MatrixXi}\hlstd{}\hlopt{(}\hlstd{n}\hlopt{,\ }\hlstd{n}\hlopt{).}\hlstd{}\hlkwd{setZero}\hlstd{}\hlopt{().}\hspace*{\fill}\\
    \hlstd{}\hlstd{\ \ \ \ \ \ \ \ \ \ \ \ \ \ \ \ \ \ \ \ \ \ }\hlstd{selfadjointView}\hlopt{$<$}\hlstd{Lower}\hlopt{$>$().}\hlstd{}\hlkwd{rankUpdate}\hlstd{}\hlopt{(}\hlstd{A}\hlopt{.}\hlstd{}\hlkwd{adjoint}\hlstd{}\hlopt{()));}\hspace*{\fill}\\
    \hlstd{MatrixXi}\hlstd{\ \ \ \ \ \ \ \ \ \ }\hlstd{}\hlkwd{AAt}\hlstd{}\hlopt{(}\hlstd{}\hlkwd{MatrixXi}\hlstd{}\hlopt{(}\hlstd{m}\hlopt{,\ }\hlstd{m}\hlopt{).}\hlstd{}\hlkwd{setZero}\hlstd{}\hlopt{().}\hspace*{\fill}\\
    \hlstd{}\hlstd{\ \ \ \ \ \ \ \ \ \ \ \ \ \ \ \ \ \ \ \ \ \ }\hlstd{selfadjointView}\hlopt{$<$}\hlstd{Lower}\hlopt{$>$().}\hlstd{}\hlkwd{rankUpdate}\hlstd{}\hlopt{(}\hlstd{A}\hlopt{));}\hspace*{\fill}\\
    \hlstd{}\hspace*{\fill}\\
    \hlkwa{return\ }\hlstd{List}\hlopt{::}\hlstd{}\hlkwd{create}\hlstd{}\hlopt{(}\hlstd{\textunderscore }\hlopt{{[}}\hlstd{}\hlstr{"crossprod(A)"}\hlstd{}\hlopt{{]}}\hlstd{\ \ }\hlopt{=\ }\hlstd{AtA}\hlopt{,}\hspace*{\fill}\\
    \hlstd{}\hlstd{\ \ \ \ \ \ \ \ \ \ \ \ \ \ \ \ \ \ \ \ }\hlstd{\textunderscore }\hlopt{{[}}\hlstd{}\hlstr{"tcrossprod(A)"}\hlstd{}\hlopt{{]}\ =\ }\hlstd{AAt}\hlopt{);}\hlstd{}\hspace*{\fill}
    \normalfont
    \normalsize
  %\end{quote}
  \caption{\textbf{crossprodCpp}: Cross-product and transposed cross-product of a single matrix}
  \label{crossprod}
\end{figure}
\begin{verbatim}
> fcprd <- cxxfunction(signature(AA = "matrix"), crossprodCpp, "RcppEigen")
> str(crp <- fcprd(A))
List of 2
 $ crossprod(A) : int [1:2, 1:2] 14 32 32 77
 $ tcrossprod(A): int [1:3, 1:3] 17 22 27 22 29 36 27 36 45
> stopifnot(all.equal(crp[[1]], crossprod(A)),
+           all.equal(crp[[2]], tcrossprod(A)))
> 
\end{verbatim}

To some, the expressions in Figure~\ref{crossprod} to construct
\code{AtA} and \code{AAt} are compact and elegant.  To others they are
hopelessly confusing.  If you find yourself in the latter group, you
just need to read the expression left to right.  So, for example, we
construct \code{AAt} by creating a general integer matrix of size
$m\times m$ (where $\bm A$ is $m\times n$), ensure that all its
elements are zero, regard it as a self-adjoint (i.e. symmetric) matrix
using the elements in the lower triangle, then add $\bm A\bm A^\prime$
to it and convert back to a general matrix form (i.e. the strict lower
triangle is copied into the strict upper triangle).

For these products one could define the symmetric matrix from either
the lower triangle or the upper triangle as the result will be
symmetrized before it is returned.

To cut down on repetition of \code{using} statements we gather them in
a character string that will be given as the \code{includes} argument
in the calls to \code{cxxfunction}.  We also define a utility
function, \code{AtA}, that returns the crossproduct matrix.


\begin{quote}
  \noindent
  \ttfamily
  \hlstd{}\hlkwa{using}\hlstd{\ \ \ }\hlkwa{}\hlstd{Eigen}\hlopt{::}\hlstd{LLT}\hlopt{;}\hspace*{\fill}\\
  \hlstd{}\hlkwa{using}\hlstd{\ \ \ }\hlkwa{}\hlstd{Eigen}\hlopt{::}\hlstd{Lower}\hlopt{;}\hspace*{\fill}\\
  \hlstd{}\hlkwa{using}\hlstd{\ \ \ }\hlkwa{}\hlstd{Eigen}\hlopt{::}\hlstd{Map}\hlopt{;}\hspace*{\fill}\\
  \hlstd{}\hlkwa{using}\hlstd{\ \ \ }\hlkwa{}\hlstd{Eigen}\hlopt{::}\hlstd{MatrixXd}\hlopt{;}\hspace*{\fill}\\
  \hlstd{}\hlkwa{using}\hlstd{\ \ \ }\hlkwa{}\hlstd{Eigen}\hlopt{::}\hlstd{MatrixXi}\hlopt{;}\hspace*{\fill}\\
  \hlstd{}\hlkwa{using}\hlstd{\ \ \ }\hlkwa{}\hlstd{Eigen}\hlopt{::}\hlstd{Upper}\hlopt{;}\hspace*{\fill}\\
  \hlstd{}\hlkwa{using}\hlstd{\ \ \ }\hlkwa{}\hlstd{Eigen}\hlopt{::}\hlstd{VectorXd}\hlopt{;}\hspace*{\fill}\\
  \hlstd{}\hlkwc{typedef\ }\hlstd{Map}\hlopt{$<$}\hlstd{MatrixXd}\hlopt{$>$}\hlstd{\ \ }\hlopt{}\hlstd{MapMatd}\hlopt{;}\hspace*{\fill}\\
  \hlstd{}\hlkwc{typedef\ }\hlstd{Map}\hlopt{$<$}\hlstd{MatrixXi}\hlopt{$>$}\hlstd{\ \ }\hlopt{}\hlstd{MapMati}\hlopt{;}\hspace*{\fill}\\
  \hlstd{}\hlkwc{typedef\ }\hlstd{Map}\hlopt{$<$}\hlstd{VectorXd}\hlopt{$>$}\hlstd{\ \ }\hlopt{}\hlstd{MapVecd}\hlopt{;}\hspace*{\fill}\\
  \hlstd{}\hlkwc{inline\ }\hlstd{MatrixXd\ }\hlkwd{AtA}\hlstd{}\hlopt{(}\hlstd{}\hlkwb{const\ }\hlstd{MapMatd}\hlopt{\&\ }\hlstd{A}\hlopt{)\ \{}\hspace*{\fill}\\
  \hlstd{}\hlstd{\ \ \ \ }\hlstd{}\hlkwb{int}\hlstd{\ \ \ \ }\hlkwb{}\hlstd{}\hlkwd{n}\hlstd{}\hlopt{(}\hlstd{A}\hlopt{.}\hlstd{}\hlkwd{cols}\hlstd{}\hlopt{());}\hspace*{\fill}\\
  \hlstd{}\hlstd{\ \ \ \ }\hlstd{}\hlkwa{return}\hlstd{\ \ \ }\hlkwa{}\hlstd{}\hlkwd{MatrixXd}\hlstd{}\hlopt{(}\hlstd{n}\hlopt{,}\hlstd{n}\hlopt{).}\hlstd{}\hlkwd{setZero}\hlstd{}\hlopt{().}\hlstd{selfadjointView}\hlopt{$<$}\hlstd{Lower}\hlopt{$>$()}\hspace*{\fill}\\
  \hlstd{}\hlstd{\ \ \ \ \ \ \ \ \ \ \ \ \ }\hlstd{}\hlopt{.}\hlstd{}\hlkwd{rankUpdate}\hlstd{}\hlopt{(}\hlstd{A}\hlopt{.}\hlstd{}\hlkwd{adjoint}\hlstd{}\hlopt{());}\hspace*{\fill}\\
  \hlstd{}\hlopt{\}}\hlstd{}\hspace*{\fill}\\
  \mbox{}
  \normalfont
  \normalsize
\end{quote}


\subsection{Cholesky decomposition of the crossprod}
\label{sec:chol}

The Cholesky decomposition of the positive-definite, symmetric matrix,
$\bm A$, can be written in several forms.  Numerical analysts define
the ``LLt'' form as the lower triangular matrix, $\bm L$, such that
$\bm A=\bm L\bm L^\prime$ and the ``LDLt'' form as a unit lower
triangular matrix $\bm L$ and a diagonal matrix $\bm D$ with positive
diagonal elements such that $\bm A=\bm L\bm D\bm L^\prime$.
Statisticians often write the decomposition as $\bm A=\bm R^\prime\bm
R$ where $\bm R$ is an upper triangular matrix.  Of course, this $\bm
R$ is simply the transpose of $\bm L$ from the ``LLt'' form.

The templated \pkg{Eigen} classes for the LLt and LDLt forms are
called \code{LLT} and \code{LDLT}.  In general, one would preserve the
objects from these classes in order to re-use them for solutions of
linear systems.  For a simple illustration, the matrix $\bm L$
from the ``LLt'' form is returned.

Because the Cholesky decomposition involves taking square roots, the internal
representation is switched to numeric matrices
\begin{verbatim}
storage.mode(A) <- "double"
\end{verbatim}
before applying the code in Figure~\ref{chol}.

\begin{figure}[htb]
  %\begin{quote}
  \noindent
  \ttfamily
  \hlstd{}\hlkwb{const}\hlstd{\ \ }\hlkwb{}\hlstd{LLT}\hlopt{$<$}\hlstd{MatrixXd}\hlopt{$>$\ }\hlstd{}\hlkwd{llt}\hlstd{}\hlopt{(}\hlstd{}\hlkwd{AtA}\hlstd{}\hlopt{(}\hlstd{as}\hlopt{$<$}\hlstd{MapMatd}\hlopt{$>$(}\hlstd{AA}\hlopt{)));}\hspace*{\fill}\\
  \hlstd{}\hlkwa{return\ }\hlstd{List}\hlopt{::}\hlstd{}\hlkwd{create}\hlstd{}\hlopt{(}\hlstd{\textunderscore }\hlopt{{[}}\hlstd{}\hlstr{"L"}\hlstd{}\hlopt{{]}\ =\ }\hlstd{}\hlkwd{MatrixXd}\hlstd{}\hlopt{(}\hlstd{llt}\hlopt{.}\hlstd{}\hlkwd{matrixL}\hlstd{}\hlopt{()),}\hspace*{\fill}\\
  \hlstd{}\hlstd{\ \ \ \ \ \ \ \ \ \ \ \ \ \ \ \ \ \ \ \ }\hlstd{\textunderscore }\hlopt{{[}}\hlstd{}\hlstr{"R"}\hlstd{}\hlopt{{]}\ =\ }\hlstd{}\hlkwd{MatrixXd}\hlstd{}\hlopt{(}\hlstd{llt}\hlopt{.}\hlstd{}\hlkwd{matrixU}\hlstd{}\hlopt{()));}\hlstd{}\hspace*{\fill}\\
  \mbox{}
  \normalfont
  \normalsize
  %\end{quote}
  \caption{\textbf{cholCpp}: Cholesky decomposition of a cross-product}
  \label{chol}
\end{figure}

\begin{verbatim}
> fchol <- cxxfunction(signature(AA = "matrix"), cholCpp, "RcppEigen", incl)
> (ll <- fchol(A))
$L
        [,1]    [,2]
[1,] 3.74166 0.00000
[2,] 8.55236 1.96396

$R
        [,1]    [,2]
[1,] 3.74166 8.55236
[2,] 0.00000 1.96396

> stopifnot(all.equal(ll[[2]], chol(crossprod(A))))
> 
\end{verbatim}



\subsection{Determinant of the cross-product matrix}
\label{sec:determinant}

The ``D-optimal'' criterion for experimental design chooses the design
that maximizes the determinant, $|\bm X^\prime\bm X|$, for the
$n\times p$ model matrix (or Jacobian matrix), $\bm X$.  The
determinant, $|\bm L|$, of the $p\times p$ lower Cholesky factor
$\bm L$, defined so that $\bm L\bm L^\prime=\bm X^\prime\bm X$, is
the product of its diagonal elements, as is the case for any
triangular matrix.  By the properties of determinants,
\begin{displaymath}
  |\bm X^\prime\bm X|=|\bm L\bm L^\prime|=|\bm L|\,|\bm L^\prime|=|\bm L|^2
\end{displaymath}

Alternatively, if using the ``LDLt'' decomposition, $\bm L\bm D\bm
L^\prime=\bm X^\prime\bm X$ where $\bm L$ is unit lower triangular and
$\bm D$ is diagonal then $|\bm X^\prime\bm X|$ is the product of the
diagonal elements of $\bm D$.  Because it is known that the diagonals of
$\bm D$ must be non-negative, one often evaluates the logarithm of the
determinant as the sum of the logarithms of the diagonal elements of
$\bm D$.  Several options are shown in Figure~\ref{cholDet}.


\begin{figure}[htb]
  %\begin{quote}
    \noindent
    \ttfamily
    \hlstd{}\hlkwa{using\ }\hlstd{Eigen}\hlopt{::}\hlstd{Lower}\hlopt{;}\hspace*{\fill}\\
    \hlstd{}\hlkwa{using\ }\hlstd{Eigen}\hlopt{::}\hlstd{Map}\hlopt{;}\hspace*{\fill}\\
    \hlstd{}\hlkwa{using\ }\hlstd{Eigen}\hlopt{::}\hlstd{MatrixXd}\hlopt{;}\hspace*{\fill}\\
    \hlstd{}\hlkwa{using\ }\hlstd{Eigen}\hlopt{::}\hlstd{VectorXd}\hlopt{;}\hspace*{\fill}\\
    \hlstd{}\hspace*{\fill}\\
    \hlkwb{const\ }\hlstd{Map}\hlopt{$<$}\hlstd{MatrixXd}\hlopt{$>$}\hlstd{\ \ \ }\hlopt{}\hlstd{}\hlkwd{A}\hlstd{}\hlopt{(}\hlstd{as}\hlopt{$<$}\hlstd{Map}\hlopt{$<$}\hlstd{MatrixXd}\hlopt{$>$\ $>$(}\hlstd{AA}\hlopt{));}\hspace*{\fill}\\
    \hlstd{}\hlkwb{const\ int}\hlstd{\ \ \ \ \ \ \ \ \ \ \ \ \ }\hlkwb{}\hlstd{}\hlkwd{n}\hlstd{}\hlopt{(}\hlstd{A}\hlopt{.}\hlstd{}\hlkwd{cols}\hlstd{}\hlopt{());}\hspace*{\fill}\\
    \hlstd{}\hlkwb{const\ }\hlstd{MatrixXd}\hlstd{\ \ \ \ \ \ }\hlstd{}\hlkwd{AtA}\hlstd{}\hlopt{(}\hlstd{}\hlkwd{MatrixXd}\hlstd{}\hlopt{(}\hlstd{n}\hlopt{,\ }\hlstd{n}\hlopt{).}\hlstd{}\hlkwd{setZero}\hlstd{}\hlopt{().}\hspace*{\fill}\\
    \hlstd{}\hlstd{\ \ \ \ \ \ \ \ \ \ \ \ \ \ \ \ \ \ \ \ \ \ \ \ }\hlstd{selfadjointView}\hlopt{$<$}\hlstd{Lower}\hlopt{$>$().}\hlstd{}\hlkwd{rankUpdate}\hlstd{}\hlopt{(}\hlstd{A}\hlopt{.}\hlstd{}\hlkwd{adjoint}\hlstd{}\hlopt{()));}\hspace*{\fill}\\
    \hlstd{}\hlkwb{const\ }\hlstd{MatrixXd}\hlstd{\ \ \ \ \ }\hlstd{}\hlkwd{Lmat}\hlstd{}\hlopt{(}\hlstd{AtA}\hlopt{.}\hlstd{}\hlkwd{llt}\hlstd{}\hlopt{().}\hlstd{}\hlkwd{matrixL}\hlstd{}\hlopt{());}\hspace*{\fill}\\
    \hlstd{}\hlkwb{const\ double}\hlstd{\ \ \ \ \ \ \ }\hlkwb{}\hlstd{}\hlkwd{detL}\hlstd{}\hlopt{(}\hlstd{Lmat}\hlopt{.}\hlstd{}\hlkwd{diagonal}\hlstd{}\hlopt{().}\hlstd{}\hlkwd{prod}\hlstd{}\hlopt{());}\hspace*{\fill}\\
    \hlstd{}\hlkwb{const\ }\hlstd{VectorXd}\hlstd{\ \ \ \ \ }\hlstd{}\hlkwd{Dvec}\hlstd{}\hlopt{(}\hlstd{AtA}\hlopt{.}\hlstd{}\hlkwd{ldlt}\hlstd{}\hlopt{().}\hlstd{}\hlkwd{vectorD}\hlstd{}\hlopt{());}\hspace*{\fill}\\
    \hlstd{}\hspace*{\fill}\\
    \hlkwa{return\ }\hlstd{List}\hlopt{::}\hlstd{}\hlkwd{create}\hlstd{}\hlopt{(}\hlstd{\textunderscore }\hlopt{{[}}\hlstd{}\hlstr{"d1"}\hlstd{}\hlopt{{]}\ =\ }\hlstd{detL\ }\hlopt{{*}\ }\hlstd{detL}\hlopt{,}\hspace*{\fill}\\
    \hlstd{}\hlstd{\ \ \ \ \ \ \ \ \ \ \ \ \ \ \ \ \ \ \ \ }\hlstd{\textunderscore }\hlopt{{[}}\hlstd{}\hlstr{"d2"}\hlstd{}\hlopt{{]}\ =\ }\hlstd{Dvec}\hlopt{.}\hlstd{}\hlkwd{prod}\hlstd{}\hlopt{(),}\hspace*{\fill}\\
    \hlstd{}\hlstd{\ \ \ \ \ \ \ \ \ \ \ \ \ \ \ \ \ \ \ \ }\hlstd{\textunderscore }\hlopt{{[}}\hlstd{}\hlstr{"ld"}\hlstd{}\hlopt{{]}\ =\ }\hlstd{Dvec}\hlopt{.}\hlstd{}\hlkwd{array}\hlstd{}\hlopt{().}\hlstd{}\hlkwd{log}\hlstd{}\hlopt{().}\hlstd{}\hlkwd{sum}\hlstd{}\hlopt{());}\hlstd{}\hspace*{\fill}
    \normalfont
    \normalsize
  %\end{quote}
  \caption{\textbf{cholDetCpp}: Determinant of a cross-product using the Cholesky decomposition}
  \label{cholDet}
\end{figure}

\begin{verbatim}
> fdet <- cxxfunction(signature(AA = "matrix"), cholDetCpp, "RcppEigen", incl)
> unlist(ll <- fdet(A))
      d1       d2       ld 
54.00000 54.00000  3.98898 
\end{verbatim}


Note the use of the \code{array()} method in the calculation of the
log-determinant.  Because the \code{log()} method applies to arrays,
not to vectors or matrices, an array from \code{Dvec} has to be created
before applying the \code{log()} method.

\section{Least squares solutions}
\label{sec:leastSquares}

A common operation in statistical computing is calculating a least
squares solution, $\widehat{\bm\beta}$, defined as
\begin{displaymath}
  \widehat{\bm\beta}=\arg\min_{\beta}\|\bm y-\bm X\bm\beta\|^2
\end{displaymath}
where the model matrix, $\bm X$, is $n\times p$ ($n\ge p$) and $\bm y$
is an $n$-dimensional response vector.  There are several ways, based
on matrix decompositions, to determine such a solution.  Earlier, two forms
of the Cholesky decomposition were discussed: ``LLt'' and
``LDLt'', which can both be used to solve for $\widehat{\bm\beta}$.  Other
decompositions that can be used are the QR decomposition, with or
without column pivoting, the singular value decomposition and the
eigendecomposition of a symmetric matrix.

Determining a least squares solution is relatively straightforward.
However, statistical computing often requires additional information,
such as the standard errors of the coefficient estimates.  Calculating
these involves evaluating the diagonal elements of $\left(\bm
  X^\prime\bm X\right)^{-1}$ and the residual sum of squares, $\|\bm
y-\bm X\widehat{\bm\beta}\|^2$.

\subsection{Least squares using the ``LLt'' Cholesky}
\label{sec:LLtLeastSquares}

\begin{figure}[tbh]
  %\begin{quote}
  \noindent
  \ttfamily
  \hlstd{}\hlkwb{const\ }\hlstd{MapMatd}\hlstd{\ \ \ \ \ \ \ \ \ }\hlstd{}\hlkwd{X}\hlstd{}\hlopt{(}\hlstd{as}\hlopt{$<$}\hlstd{MapMatd}\hlopt{$>$(}\hlstd{XX}\hlopt{));}\hspace*{\fill}\\
  \hlstd{}\hlkwb{const\ }\hlstd{MapVecd}\hlstd{\ \ \ \ \ \ \ \ \ }\hlstd{}\hlkwd{y}\hlstd{}\hlopt{(}\hlstd{as}\hlopt{$<$}\hlstd{MapVecd}\hlopt{$>$(}\hlstd{yy}\hlopt{));}\hspace*{\fill}\\
  \hlstd{}\hlkwb{const\ int}\hlstd{\ \ \ \ \ \ \ \ \ \ \ \ \ }\hlkwb{}\hlstd{}\hlkwd{n}\hlstd{}\hlopt{(}\hlstd{X}\hlopt{.}\hlstd{}\hlkwd{rows}\hlstd{}\hlopt{()),\ }\hlstd{}\hlkwd{p}\hlstd{}\hlopt{(}\hlstd{X}\hlopt{.}\hlstd{}\hlkwd{cols}\hlstd{}\hlopt{());}\hspace*{\fill}\\
  \hlstd{}\hlkwb{const\ }\hlstd{LLT}\hlopt{$<$}\hlstd{MatrixXd}\hlopt{$>$\ }\hlstd{}\hlkwd{llt}\hlstd{}\hlopt{(}\hlstd{}\hlkwd{AtA}\hlstd{}\hlopt{(}\hlstd{X}\hlopt{));}\hspace*{\fill}\\
  \hlstd{}\hlkwb{const\ }\hlstd{VectorXd}\hlstd{\ \ }\hlstd{}\hlkwd{betahat}\hlstd{}\hlopt{(}\hlstd{llt}\hlopt{.}\hlstd{}\hlkwd{solve}\hlstd{}\hlopt{(}\hlstd{X}\hlopt{.}\hlstd{}\hlkwd{adjoint}\hlstd{}\hlopt{()\ {*}\ }\hlstd{y}\hlopt{));}\hspace*{\fill}\\
  \hlstd{}\hlkwb{const\ }\hlstd{VectorXd}\hlstd{\ \ \ }\hlstd{}\hlkwd{fitted}\hlstd{}\hlopt{(}\hlstd{X\ }\hlopt{{*}\ }\hlstd{betahat}\hlopt{);}\hspace*{\fill}\\
  \hlstd{}\hlkwb{const\ }\hlstd{VectorXd}\hlstd{\ \ \ \ }\hlstd{}\hlkwd{resid}\hlstd{}\hlopt{(}\hlstd{y\ }\hlopt{{-}\ }\hlstd{fitted}\hlopt{);}\hspace*{\fill}\\
  \hlstd{}\hlkwb{const\ int}\hlstd{\ \ \ \ \ \ \ \ \ \ \ \ }\hlkwb{}\hlstd{}\hlkwd{df}\hlstd{}\hlopt{(}\hlstd{n\ }\hlopt{{-}\ }\hlstd{p}\hlopt{);}\hspace*{\fill}\\
  \hlstd{}\hlkwb{const\ double}\hlstd{\ \ \ \ \ \ \ \ \ \ }\hlkwb{}\hlstd{}\hlkwd{s}\hlstd{}\hlopt{(}\hlstd{resid}\hlopt{.}\hlstd{}\hlkwd{norm}\hlstd{}\hlopt{()\ /\ }\hlstd{std}\hlopt{::}\hlstd{}\hlkwd{sqrt}\hlstd{}\hlopt{(}\hlstd{}\hlkwb{double}\hlstd{}\hlopt{(}\hlstd{df}\hlopt{)));}\hspace*{\fill}\\
  \hlstd{}\hlkwb{const\ }\hlstd{VectorXd}\hlstd{\ \ \ \ \ \ \ }\hlstd{}\hlkwd{se}\hlstd{}\hlopt{(}\hlstd{s\ }\hlopt{{*}\ }\hlstd{llt}\hlopt{.}\hlstd{}\hlkwd{matrixL}\hlstd{}\hlopt{().}\hlstd{}\hlkwd{solve}\hlstd{}\hlopt{(}\hlstd{MatrixXd}\hlopt{::}\hlstd{}\hlkwd{Identity}\hlstd{}\hlopt{(}\hlstd{p}\hlopt{,\ }\hlstd{p}\hlopt{))}\hspace*{\fill}\\
  \hlstd{}\hlstd{\ \ \ \ \ \ \ \ \ \ \ \ \ \ \ \ \ \ \ \ \ \ \ \ }\hlstd{}\hlopt{.}\hlstd{}\hlkwd{colwise}\hlstd{}\hlopt{().}\hlstd{}\hlkwd{norm}\hlstd{}\hlopt{());}\hspace*{\fill}\\
  \hlstd{}\hlkwa{return}\hlstd{\ \ \ \ \ }\hlkwa{}\hlstd{List}\hlopt{::}\hlstd{}\hlkwd{create}\hlstd{}\hlopt{(}\hlstd{\textunderscore }\hlopt{{[}}\hlstd{}\hlstr{"coefficients"}\hlstd{}\hlopt{{]}}\hlstd{\ \ \ }\hlopt{=\ }\hlstd{betahat}\hlopt{,}\hspace*{\fill}\\
  \hlstd{}\hlstd{\ \ \ \ \ \ \ \ \ \ \ \ \ \ \ \ \ \ \ \ \ \ \ \ }\hlstd{\textunderscore }\hlopt{{[}}\hlstd{}\hlstr{"fitted.values"}\hlstd{}\hlopt{{]}}\hlstd{\ \ }\hlopt{=\ }\hlstd{fitted}\hlopt{,}\hspace*{\fill}\\
  \hlstd{}\hlstd{\ \ \ \ \ \ \ \ \ \ \ \ \ \ \ \ \ \ \ \ \ \ \ \ }\hlstd{\textunderscore }\hlopt{{[}}\hlstd{}\hlstr{"residuals"}\hlstd{}\hlopt{{]}}\hlstd{\ \ \ \ \ \ }\hlopt{=\ }\hlstd{resid}\hlopt{,}\hspace*{\fill}\\
  \hlstd{}\hlstd{\ \ \ \ \ \ \ \ \ \ \ \ \ \ \ \ \ \ \ \ \ \ \ \ }\hlstd{\textunderscore }\hlopt{{[}}\hlstd{}\hlstr{"s"}\hlstd{}\hlopt{{]}}\hlstd{\ \ \ \ \ \ \ \ \ \ \ \ \ \ }\hlopt{=\ }\hlstd{s}\hlopt{,}\hspace*{\fill}\\
  \hlstd{}\hlstd{\ \ \ \ \ \ \ \ \ \ \ \ \ \ \ \ \ \ \ \ \ \ \ \ }\hlstd{\textunderscore }\hlopt{{[}}\hlstd{}\hlstr{"df.residual"}\hlstd{}\hlopt{{]}}\hlstd{\ \ \ \ }\hlopt{=\ }\hlstd{df}\hlopt{,}\hspace*{\fill}\\
  \hlstd{}\hlstd{\ \ \ \ \ \ \ \ \ \ \ \ \ \ \ \ \ \ \ \ \ \ \ \ }\hlstd{\textunderscore }\hlopt{{[}}\hlstd{}\hlstr{"rank"}\hlstd{}\hlopt{{]}}\hlstd{\ \ \ \ \ \ \ \ \ \ \ }\hlopt{=\ }\hlstd{p}\hlopt{,}\hspace*{\fill}\\
  \hlstd{}\hlstd{\ \ \ \ \ \ \ \ \ \ \ \ \ \ \ \ \ \ \ \ \ \ \ \ }\hlstd{\textunderscore }\hlopt{{[}}\hlstd{}\hlstr{"Std.\ Error"}\hlstd{}\hlopt{{]}}\hlstd{\ \ \ \ \ }\hlopt{=\ }\hlstd{se}\hlopt{);}\hlstd{}\hspace*{\fill}\\
  \mbox{}
  \normalfont
  \normalsize
  %\end{quote}
  \caption{\textbf{lltLSCpp}: Least squares using the Cholesky decomposition}
  \label{lltLS}
\end{figure}
Figure~\ref{lltLS} shows a calculation of the least squares coefficient estimates
(\code{betahat}) and the standard errors (\code{se}) through an
``LLt'' Cholesky decomposition of the crossproduct of the model
matrix, $\bm X$.  Next, the results from this calculation are compared
to those from the \code{lm.fit} function in \proglang{R}
(\code{lm.fit} is the workhorse function called by \code{lm} once the
model matrix and response have been evaluated).
\begin{verbatim}
> lltLS <- cxxfunction(signature(XX = "matrix", yy = "numeric"), 
+                      lltLSCpp, "RcppEigen", incl)
> data(trees, package="datasets")
> str(lltFit <- with(trees, lltLS(cbind(1, log(Girth)), log(Volume))))
List of 7
 $ coefficients : num [1:2] -2.35 2.2
 $ fitted.values: num [1:31] 2.3 2.38 2.43 2.82 2.86 ...
 $ residuals    : num [1:31] 0.0298 -0.0483 -0.1087 -0.0223 0.0727 ...
 $ s            : num 0.115
 $ df.residual  : int 29
 $ rank         : int 2
 $ Std. Error   : num [1:2] 0.2307 0.0898
> str(lmFit <- with(trees, lm.fit(cbind(1, log(Girth)), log(Volume))))
List of 8
 $ coefficients : Named num [1:2] -2.35 2.2
  ..- attr(*, "names")= chr [1:2] "x1" "x2"
 $ residuals    : num [1:31] 0.0298 -0.0483 -0.1087 -0.0223 0.0727 ...
 $ effects      : Named num [1:31] -18.2218 2.8152 -0.1029 -0.0223 0.0721 ...
  ..- attr(*, "names")= chr [1:31] "x1" "x2" "" "" ...
 $ rank         : int 2
 $ fitted.values: num [1:31] 2.3 2.38 2.43 2.82 2.86 ...
 $ assign       : NULL
 $ qr           :List of 5
  ..$ qr   : num [1:31, 1:2] -5.57 0.18 0.18 0.18 0.18 ...
  ..$ qraux: num [1:2] 1.18 1.26
  ..$ pivot: int [1:2] 1 2
  ..$ tol  : num 1e-07
  ..$ rank : int 2
  ..- attr(*, "class")= chr "qr"
 $ df.residual  : int 29
> for (nm in c("coefficients", "residuals", "fitted.values", "rank"))
+     stopifnot(all.equal(lltFit[[nm]], unname(lmFit[[nm]])))
> stopifnot(all.equal(lltFit[["Std. Error"]],
+                     unname(coef(summary(lm(log(Volume) ~ log(Girth), trees)))[,2])))
> 
\end{verbatim}


There are several aspects of the \proglang{C++} code in
Figure~\ref{lltLS} worth mentioning.  The \code{solve} method for the
\code{LLT} object evaluates, in this case, $\left(\bm X^\prime\bm
  X\right)^{-1}\bm X^\prime\bm y$ but without actually evaluating the
inverse.  The calculation of the residuals, $\bm y-\widehat{\bm y}$,
can be written, as in \proglang{R}, as \code{y - fitted}. (But note
that \pkg{Eigen} classes do not have a ``recycling rule as in
\proglang{R}.  That is, the two vector operands must have the same
length.)  The \code{norm()} method evaluates the square root of the
sum of squares of the elements of a vector.  Although one does not
explicitly evaluate $\left(\bm X^\prime\bm X\right)^{-1}$ one does
evaluate $\bm L^{-1}$ to obtain the standard errors.  Note also the
use of the \code{colwise()} method in the evaluation of the standard
errors.  It applies a method to the columns of a matrix, returning a
vector.  The \pkg{Eigen} \code{colwise()} and \code{rowwise()} methods
are similar in effect to the \code{apply} function in \proglang{R}.

In the descriptions of other methods for solving least squares
problems, much of the code parallels that shown in
Figure~\ref{lltLS}.  The redundant parts are omitted, and only
the evaluation of the coefficients, the rank and the standard errors is shown.
Actually, the standard errors are calculated only up to the scalar
multiple of $s$, the residual standard error, in these code fragments.
The calculation of the residuals and $s$ and the scaling of the
coefficient standard errors is the same for all methods.  (See the
files \code{fastLm.h} and \code{fastLm.cpp} in the \pkg{RcppEigen}
source package for details.)


\subsection{Least squares using the unpivoted QR decomposition}
\label{sec:QR}

A QR decomposition has the form
\begin{displaymath}
  \bm X=\bm Q\bm R=\bm Q_1\bm R_1
\end{displaymath}
where $\bm Q$ is an $n\times n$ orthogonal matrix, which means that
$\bm Q^\prime\bm Q=\bm Q\bm Q^\prime=\bm I_n$, and the $n\times p$
matrix $\bm R$ is zero below the main diagonal.  The $n\times p$
matrix $\bm Q_1$ is the first $p$ columns of $\bm Q$ and the $p\times
p$ upper triangular matrix $\bm R_1$ is the top $p$ rows of $\bm R$.
There are three \pkg{Eigen} classes for the QR decomposition:
\code{HouseholderQR} provides the basic QR decomposition using
Householder transformations, \code{ColPivHouseholderQR} incorporates
column pivots and \code{FullPivHouseholderQR} incorporates both row
and column pivots.

Figure~\ref{QRLS} shows a least squares solution using the unpivoted
QR decomposition.
% using Eigen::HouseholderQR;

% const HouseholderQR<MatrixXd> QR(X);
% const VectorXd           betahat(QR.solve(y));
% const VectorXd            fitted(X * betahat);
% const int                     df(n - p);
% const VectorXd                se(QR.matrixQR().topRows(p).
%                                  triangularView<Upper>().
%                                  solve(MatrixXd::Identity(p,p)).
%                                  rowwise().norm());
\begin{figure}[htb]
  %\begin{quote}
    \noindent
    \ttfamily
    \hlstd{}\hlkwa{using\ }\hlstd{Eigen}\hlopt{::}\hlstd{HouseholderQR}\hlopt{;}\hspace*{\fill}\\
    \hlstd{}\hspace*{\fill}\\
    \hlkwb{const\ }\hlstd{HouseholderQR}\hlopt{$<$}\hlstd{MatrixXd}\hlopt{$>$\ }\hlstd{}\hlkwd{QR}\hlstd{}\hlopt{(}\hlstd{X}\hlopt{);}\hspace*{\fill}\\
    \hlstd{}\hlkwb{const\ }\hlstd{VectorXd}\hlstd{\ \ \ \ \ \ \ \ \ \ \ }\hlstd{}\hlkwd{betahat}\hlstd{}\hlopt{(}\hlstd{QR}\hlopt{.}\hlstd{}\hlkwd{solve}\hlstd{}\hlopt{(}\hlstd{y}\hlopt{));}\hspace*{\fill}\\
    \hlstd{}\hlkwb{const\ }\hlstd{VectorXd}\hlstd{\ \ \ \ \ \ \ \ \ \ \ \ }\hlstd{}\hlkwd{fitted}\hlstd{}\hlopt{(}\hlstd{X\ }\hlopt{{*}\ }\hlstd{betahat}\hlopt{);}\hspace*{\fill}\\
    \hlstd{}\hlkwb{const\ int}\hlstd{\ \ \ \ \ \ \ \ \ \ \ \ \ \ \ \ \ \ \ \ \ }\hlkwb{}\hlstd{}\hlkwd{df}\hlstd{}\hlopt{(}\hlstd{n\ }\hlopt{{-}\ }\hlstd{p}\hlopt{);}\hspace*{\fill}\\
    \hlstd{}\hlkwb{const\ }\hlstd{VectorXd}\hlstd{\ \ \ \ \ \ \ \ \ \ \ \ \ \ \ \ }\hlstd{}\hlkwd{se}\hlstd{}\hlopt{(}\hlstd{QR}\hlopt{.}\hlstd{}\hlkwd{matrixQR}\hlstd{}\hlopt{().}\hlstd{}\hlkwd{topRows}\hlstd{}\hlopt{(}\hlstd{p}\hlopt{).}\hspace*{\fill}\\
    \hlstd{}\hlstd{\ \ \ \ \ \ \ \ \ \ \ \ \ \ \ \ \ \ \ \ \ \ \ \ \ \ \ \ \ \ \ \ \ }\hlstd{triangularView}\hlopt{$<$}\hlstd{Upper}\hlopt{$>$().}\hspace*{\fill}\\
    \hlstd{}\hlstd{\ \ \ \ \ \ \ \ \ \ \ \ \ \ \ \ \ \ \ \ \ \ \ \ \ \ \ \ \ \ \ \ \ }\hlstd{}\hlkwd{solve}\hlstd{}\hlopt{(}\hlstd{MatrixXd}\hlopt{::}\hlstd{}\hlkwd{Identity}\hlstd{}\hlopt{(}\hlstd{p}\hlopt{,}\hlstd{p}\hlopt{)).}\hspace*{\fill}\\
    \hlstd{}\hlstd{\ \ \ \ \ \ \ \ \ \ \ \ \ \ \ \ \ \ \ \ \ \ \ \ \ \ \ \ \ \ \ \ \ }\hlstd{}\hlkwd{rowwise}\hlstd{}\hlopt{().}\hlstd{}\hlkwd{norm}\hlstd{}\hlopt{());}\hlstd{}\hspace*{\fill}
    \normalfont
    \normalsize
    \caption{\textbf{QRLSCpp}: Least squares using the unpivoted QR decomposition}
    \label{QRLS}
  %\end{quote}
\end{figure}
The calculations in Figure~\ref{QRLS} are quite similar to those in
Figure~\ref{lltLS}.  In fact, if one had extracted the upper
triangular factor (the \code{matrixU()} method) from the \code{LLT}
object in Figure~\ref{lltLS}, the rest of the code would be nearly
identical.


\subsection{Handling the rank-deficient case}
\label{sec:rankdeficient}

One important consideration when determining least squares solutions
is whether $\rank(\bm X)$ is $p$, a situation described by saying
that $\bm X$ has ``full column rank''.   When $\bm X$ does not have
full column rank it is said to be ``rank deficient''.

Although the theoretical rank of a matrix is well-defined, its
evaluation in practice is not.  At best one can compute an effective
rank according to some tolerance.  Decompositions that allow to
estimation of the rank of the matrix in this way are said to be
``rank-revealing''.

Because the \code{model.matrix} function in \proglang{R} does a
considerable amount of symbolic analysis behind the scenes, one usually
ends up with full-rank model matrices.  The common cases of
rank-deficiency, such as incorporating both a constant term and a full
set of indicators columns for the levels of a factor, are eliminated.
Other, more subtle, situations will not be detected at this stage,
however.  A simple example occurs when there is a ``missing cell'' in a
two-way layout and the interaction of the two factors is included in
the model.

\begin{verbatim}
> dd <- data.frame(f1 = gl(4, 6, labels = LETTERS[1:4]),
+                  f2 = gl(3, 2, labels = letters[1:3]))[-(7:8), ]
> xtabs(~ f2 + f1, dd)                    # one missing cell
   f1
f2  A B C D
  a 2 0 2 2
  b 2 2 2 2
  c 2 2 2 2
> mm <- model.matrix(~ f1 * f2, dd)
> kappa(mm)         # large condition number, indicating rank deficiency
[1] 4.30923e+16
> rcond(mm)         # alternative evaluation, the reciprocal condition number
[1] 2.3206e-17
> (c(rank=qr(mm)$rank, p=ncol(mm))) # rank as computed in R's qr function
rank    p 
  11   12 
> set.seed(1)
> dd$y <- mm %*% seq_len(ncol(mm)) + rnorm(nrow(mm), sd = 0.1)
> fm1 <- lm(y ~ f1 * f2, dd)
> writeLines(capture.output(print(summary(fm1), signif.stars=FALSE))[9:22])
Coefficients: (1 not defined because of singularities)
            Estimate Std. Error t value Pr(>|t|)
(Intercept)   0.9779     0.0582    16.8  3.4e-09
f1B          12.0381     0.0823   146.3  < 2e-16
f1C           3.1172     0.0823    37.9  5.2e-13
f1D           4.0685     0.0823    49.5  2.8e-14
f2b           5.0601     0.0823    61.5  2.6e-15
f2c           5.9976     0.0823    72.9  4.0e-16
f1B:f2b      -3.0148     0.1163   -25.9  3.3e-11
f1C:f2b       7.7030     0.1163    66.2  1.2e-15
f1D:f2b       8.9643     0.1163    77.1  < 2e-16
f1B:f2c           NA         NA      NA       NA
f1C:f2c      10.9613     0.1163    94.2  < 2e-16
f1D:f2c      12.0411     0.1163   103.5  < 2e-16
> 
\end{verbatim}

The \code{lm} function for fitting linear models in \proglang{R} uses
a rank-revealing form of the QR decomposition.  When the model matrix
is determined to be rank deficient, according to the threshold used in
\proglang{R}'s QR decomposition, the model matrix is reduced to
$\rank{(\bm X)}$ columns by pivoting selected columns (those that are
apparently linearly dependent on columns to their left) to the right
hand side of the matrix.  A solution for this reduced model matrix is
determined and the coefficients and standard errors for the redundant
columns are flagged as missing.

An alternative approach is to evaluate the ``pseudo-inverse'' of $\bm
X$ from the singular value decomposition (SVD) of $\bm X$ or the
eigendecomposition of $\bm X^\prime\bm X$.  The SVD is of the form
\begin{displaymath}
  \bm X=\bm U\bm D\bm V^\prime=\bm U_1\bm D_1\bm V^\prime
\end{displaymath}
where $\bm U$ is an orthogonal $n\times n$ matrix and $\bm U_1$ is its
leftmost $p$ columns, $\bm D$ is $n\times p$ and zero off the main
diagonal so that $\bm D_1$ is a $p\times p$ diagonal matrix with
non-decreasing non-negative diagonal elements, and $\bm V$ is a $p\times
p$ orthogonal matrix.  The pseudo-inverse of $\bm D_1$, written $\bm
D_1^+$ is a $p\times p$ diagonal matrix whose first $r=\rank(\bm X)$
diagonal elements are the inverses of the corresponding diagonal
elements of $\bm D_1$ and whose last $p-r$ diagonal elements are zero.

The tolerance for determining if an element of the diagonal of $\bm D$
is considered to be (effectively) zero is a multiple of the largest
singular value (i.e. the $(1,1)$ element of $\bm D$).

The pseudo-inverse, $\bm X^+$, of $\bm X$ is defined as
\begin{displaymath}
  \bm X^+=\bm V\bm D_1^+\bm U_1^\prime .
\end{displaymath}

In Figure~\ref{Dplus} a utility function, \code{Dplus}, is defined to
return the diagonal of the pseudo-inverse, $\bm D_1^+$, as an array,
given the singular values (the diagonal of $\bm D$) as an array.

\begin{figure}[htb]
  %\begin{quote}
  \noindent
  \ttfamily
  \hlstd{}\hlkwc{inline\ }\hlstd{ArrayXd\ }\hlkwd{Dplus}\hlstd{}\hlopt{(}\hlstd{}\hlkwb{const\ }\hlstd{ArrayXd}\hlopt{\&\ }\hlstd{d}\hlopt{)\ \{}\hspace*{\fill}\\
  \hlstd{}\hlstd{\ \ \ \ }\hlstd{ArrayXd}\hlstd{\ \ \ }\hlstd{}\hlkwd{di}\hlstd{}\hlopt{(}\hlstd{d}\hlopt{.}\hlstd{}\hlkwd{size}\hlstd{}\hlopt{());}\hspace*{\fill}\\
  \hlstd{}\hlstd{\ \ \ \ }\hlstd{}\hlkwb{double}\hlstd{\ \ }\hlkwb{}\hlstd{}\hlkwd{comp}\hlstd{}\hlopt{(}\hlstd{d}\hlopt{.}\hlstd{}\hlkwd{maxCoeff}\hlstd{}\hlopt{()\ {*}\ }\hlstd{}\hlkwd{threshold}\hlstd{}\hlopt{());}\hspace*{\fill}\\
  \hlstd{}\hlstd{\ \ \ \ }\hlstd{}\hlkwa{for\ }\hlstd{}\hlopt{(}\hlstd{}\hlkwb{int\ }\hlstd{j\ }\hlopt{=\ }\hlstd{}\hlnum{0}\hlstd{}\hlopt{;\ }\hlstd{j\ }\hlopt{$<$\ }\hlstd{d}\hlopt{.}\hlstd{}\hlkwd{size}\hlstd{}\hlopt{();\ ++}\hlstd{j}\hlopt{)\ }\hlstd{di}\hlopt{{[}}\hlstd{j}\hlopt{{]}\ =\ (}\hlstd{d}\hlopt{{[}}\hlstd{j}\hlopt{{]}\ $<$\ }\hlstd{comp}\hlopt{)\ }\hlstd{?\ }\hlnum{0}\hlstd{}\hlopt{.\ :\ }\hlstd{}\hlnum{1}\hlstd{}\hlopt{./}\hlstd{d}\hlopt{{[}}\hlstd{j}\hlopt{{]};}\hspace*{\fill}\\
  \hlstd{}\hlstd{\ \ \ \ }\hlstd{}\hlkwa{return\ }\hlstd{di}\hlopt{;}\hspace*{\fill}\\
  \hlstd{}\hlopt{\}}\hlstd{}\hspace*{\fill}\\
  \mbox{}
  \normalfont
  \normalsize
  %\end{quote}
  \caption{\textbf{plusCpp}: Create the diagonal $\bm di$ of the pseudo-inverse, $\bm D_1^+$, from the array of singular values, $\bm d$}
  \label{Dplus}
\end{figure}
% inline ArrayXd Dplus(const ArrayXd& d) {
%     ArrayXd   di(d.size());
%     double  comp(d.maxCoeff() * threshold());
%     for (int j = 0; j < d.size(); ++j) di[j] = (d[j] < comp) ? 0. : 1./d[j];
%     return di;
% }

Calculation of the maximum element of $\bm d$ (the method is called
\code{.maxCoeff()}) and the use of a \code{threshold()} function
provides greater generality of the function.  It can be used on the
eigenvalues of $\bm X^\prime\bm X$, as shown in
Sec.~\ref{sec:eigendecomp}, even though these are returned in
increasing order, as opposed to the decreasing order of the singular
values.

\subsection{Least squares using the SVD}
\label{sec:SVDls}

With these definitions the code for least squares using the singular
value decomposition can be written as in Figure~\ref{SVDLS}.
\begin{figure}[htb]
  %\begin{quote}
    \noindent
    \ttfamily
    \hlstd{}\hlkwa{using\ }\hlstd{Eigen}\hlopt{::}\hlstd{JacobiSVD}\hlopt{;}\hspace*{\fill}\\
    \hlstd{}\hspace*{\fill}\\
    \hlkwb{const\ }\hlstd{JacobiSVD}\hlopt{$<$}\hlstd{MatrixXd}\hlopt{$>$}\hspace*{\fill}\\
    \hlstd{}\hlstd{\ \ \ \ \ \ \ \ \ \ \ \ \ \ \ \ }\hlstd{}\hlkwd{UDV}\hlstd{}\hlopt{(}\hlstd{X}\hlopt{.}\hlstd{}\hlkwd{jacobiSvd}\hlstd{}\hlopt{(}\hlstd{Eigen}\hlopt{::}\hlstd{ComputeThinU}\hlopt{\textbar }\hlstd{Eigen}\hlopt{::}\hlstd{ComputeThinV}\hlopt{));}\hspace*{\fill}\\
    \hlstd{}\hlkwb{const\ }\hlstd{ArrayXd}\hlstd{\ \ \ \ \ \ \ \ \ \ \ \ \ \ \ }\hlstd{}\hlkwd{D}\hlstd{}\hlopt{(}\hlstd{UDV}\hlopt{.}\hlstd{}\hlkwd{singularValues}\hlstd{}\hlopt{());}\hspace*{\fill}\\
    \hlstd{}\hlkwb{const\ int}\hlstd{\ \ \ \ \ \ \ \ \ \ \ \ \ \ \ \ \ \ \ }\hlkwb{}\hlstd{}\hlkwd{r}\hlstd{}\hlopt{((}\hlstd{D\ }\hlopt{$>$\ }\hlstd{D}\hlopt{{[}}\hlstd{}\hlnum{0}\hlstd{}\hlopt{{]}\ {*}\ }\hlstd{}\hlkwd{threshold}\hlstd{}\hlopt{()).}\hlstd{}\hlkwd{count}\hlstd{}\hlopt{());}\hspace*{\fill}\\
    \hlstd{}\hlkwb{const\ }\hlstd{MatrixXd}\hlstd{\ \ \ \ \ \ \ \ \ \ \ \ }\hlstd{}\hlkwd{VDp}\hlstd{}\hlopt{(}\hlstd{UDV}\hlopt{.}\hlstd{}\hlkwd{matrixV}\hlstd{}\hlopt{()\ {*}\ }\hlstd{}\hlkwd{Dplus}\hlstd{}\hlopt{(}\hlstd{D}\hlopt{,\ }\hlstd{r}\hlopt{));}\hspace*{\fill}\\
    \hlstd{}\hlkwb{const\ }\hlstd{VectorXd}\hlstd{\ \ \ \ \ \ \ \ }\hlstd{}\hlkwd{betahat}\hlstd{}\hlopt{(}\hlstd{VDp\ }\hlopt{{*}\ }\hlstd{UDV}\hlopt{.}\hlstd{}\hlkwd{matrixU}\hlstd{}\hlopt{().}\hlstd{}\hlkwd{adjoint}\hlstd{}\hlopt{()\ {*}\ }\hlstd{y}\hlopt{);}\hspace*{\fill}\\
    \hlstd{}\hlkwb{const\ int}\hlstd{\ \ \ \ \ \ \ \ \ \ \ \ \ \ \ \ \ \ }\hlkwb{}\hlstd{}\hlkwd{df}\hlstd{}\hlopt{(}\hlstd{n\ }\hlopt{{-}\ }\hlstd{r}\hlopt{);}\hspace*{\fill}\\
    \hlstd{}\hlkwb{const\ }\hlstd{VectorXd}\hlstd{\ \ \ \ \ \ \ \ \ \ \ \ \ }\hlstd{}\hlkwd{se}\hlstd{}\hlopt{(}\hlstd{s\ }\hlopt{{*}\ }\hlstd{VDp}\hlopt{.}\hlstd{}\hlkwd{rowwise}\hlstd{}\hlopt{().}\hlstd{}\hlkwd{norm}\hlstd{}\hlopt{());}\hlstd{}\hspace*{\fill}
    \normalfont
    \normalsize
  %\end{quote}
  \caption{\textbf{SVDLSCpp}: Least squares using the SVD}
  \label{SVDLS}
\end{figure}
% using Eigen::JacobiSVD;

% const JacobiSVD<MatrixXd>
%                 UDV(X.jacobiSvd(Eigen::ComputeThinU|Eigen::ComputeThinV));
% const ArrayXd               D(UDV.singularValues());
% const int                   r((D > D[0] * threshold()).count());
% const MatrixXd            VDp(UDV.matrixV() * Dplus(D, r));
% const VectorXd        betahat(VDp * UDV.matrixU().adjoint() * y);
% const int                  df(n - r);
% const VectorXd             se(s * VDp.rowwise().norm());
In the rank-deficient case this code will produce a complete set of
coefficients and their standard errors.  It is up to the user to note
that the rank is less than $p$, the number of columns in $\bm X$, and
hence that the estimated coefficients are just one of an infinite
number of coefficient vectors that produce the same fitted values.  It
happens that this solution is the minimum norm solution.

The standard errors of the coefficient estimates in the rank-deficient
case must be interpreted carefully.  The solution with one or more missing
coefficients, as returned by the \code{lm.fit} function in
\proglang{R} and by the column-pivoted QR decomposition described in
Section~\ref{sec:colPivQR}, does not provide standard errors for the
missing coefficients.  That is, both the coefficient and its standard
error are returned as \code{NA} because the least squares solution is
performed on a reduced model matrix.  It is also true that the
solution returned by the SVD method is with respect to a reduced model
matrix but the $p$ coefficient estimates and their $p$ standard errors
don't show this.  They are, in fact, linear combinations of a set of
$r$ coefficient estimates and their standard errors.

\subsection{Least squares using the eigendecomposition}
\label{sec:eigendecomp}

The eigendecomposition of $\bm X^\prime\bm X$ is defined as
\begin{displaymath}
  \bm X^\prime\bm X=\bm V\bm\Lambda\bm V^\prime
\end{displaymath}
where $\bm V$, the matrix of eigenvectors, is a $p\times p$ orthogonal
matrix and $\bm\Lambda$ is a $p\times p$ diagonal matrix with
non-increasing, non-negative diagonal elements, called the eigenvalues
of $\bm X^\prime\bm X$.  When the eigenvalues are distinct, this $\bm
V$ is the same as that in the SVD and the eigenvalues of $\bm
X^\prime\bm X$ are the squares of the singular values of $\bm X$.

With these definitions one can adapt much of the code from the SVD
method for the eigendecomposition, as shown in Figure~\ref{SymmEigLS}.
\begin{figure}[htb]
  %\begin{quote}
    \noindent
    \ttfamily
    \hlstd{}\hlkwa{using\ }\hlstd{Eigen}\hlopt{::}\hlstd{SelfAdjointEigenSolver}\hlopt{;}\hspace*{\fill}\\
    \hlstd{}\hspace*{\fill}\\
    \hlkwb{const\ }\hlstd{SelfAdjointEigenSolver}\hlopt{$<$}\hlstd{MatrixXd}\hlopt{$>$}\hspace*{\fill}\\
    \hlstd{}\hlstd{\ \ \ \ \ \ \ \ \ \ \ \ \ \ \ \ \ \ \ \ \ \ \ \ \ \ }\hlstd{}\hlkwd{VLV}\hlstd{}\hlopt{(}\hlstd{}\hlkwd{MatrixXd}\hlstd{}\hlopt{(}\hlstd{p}\hlopt{,\ }\hlstd{p}\hlopt{).}\hlstd{}\hlkwd{setZero}\hlstd{}\hlopt{()}\hspace*{\fill}\\
    \hlstd{}\hlstd{\ \ \ \ \ \ \ \ \ \ \ \ \ \ \ \ \ \ \ \ \ \ \ \ \ \ \ \ \ \ \ \ \ \ \ \ \ \ \ \ \ \ \ \ }\hlstd{}\hlopt{.}\hlstd{selfadjointView}\hlopt{$<$}\hlstd{Lower}\hlopt{$>$}\hspace*{\fill}\\
    \hlstd{}\hlstd{\ \ \ \ \ \ \ \ \ \ \ \ \ \ \ \ \ \ \ \ \ \ \ \ \ \ \ \ \ \ \ \ \ \ \ \ \ \ \ \ \ \ \ \ }\hlstd{}\hlopt{.}\hlstd{}\hlkwd{rankUpdate}\hlstd{}\hlopt{(}\hlstd{X}\hlopt{.}\hlstd{}\hlkwd{adjoint}\hlstd{}\hlopt{()));}\hspace*{\fill}\\
    \hlstd{}\hlkwb{const\ }\hlstd{ArrayXd}\hlstd{\ \ \ \ \ \ \ \ \ \ \ \ \ \ \ }\hlstd{}\hlkwd{D}\hlstd{}\hlopt{(}\hlstd{eig}\hlopt{.}\hlstd{}\hlkwd{eigenvalues}\hlstd{}\hlopt{());}\hspace*{\fill}\\
    \hlstd{}\hlkwb{const\ int}\hlstd{\ \ \ \ \ \ \ \ \ \ \ \ \ \ \ \ \ \ \ }\hlkwb{}\hlstd{}\hlkwd{r}\hlstd{}\hlopt{((}\hlstd{D\ }\hlopt{$>$\ }\hlstd{D}\hlopt{{[}}\hlstd{p\ }\hlopt{{-}\ }\hlstd{}\hlnum{1}\hlstd{}\hlopt{{]}\ {*}\ }\hlstd{}\hlkwd{threshold}\hlstd{}\hlopt{()).}\hlstd{}\hlkwd{count}\hlstd{}\hlopt{());}\hspace*{\fill}\\
    \hlstd{}\hlkwb{const\ }\hlstd{MatrixXd}\hlstd{\ \ \ \ \ \ \ \ \ \ \ \ }\hlstd{}\hlkwd{VDp}\hlstd{}\hlopt{(}\hlstd{VLV}\hlopt{.}\hlstd{}\hlkwd{eigenvectors}\hlstd{}\hlopt{()\ {*}\ }\hlstd{}\hlkwd{Dplus}\hlstd{}\hlopt{(}\hlstd{D}\hlopt{.}\hlstd{}\hlkwd{sqrt}\hlstd{}\hlopt{(),}\hlstd{r}\hlopt{,}\hlstd{}\hlkwa{true}\hlstd{}\hlopt{));}\hspace*{\fill}\\
    \hlstd{}\hlkwb{const\ }\hlstd{VectorXd}\hlstd{\ \ \ \ \ \ \ \ }\hlstd{}\hlkwd{betahat}\hlstd{}\hlopt{(}\hlstd{VDp\ }\hlopt{{*}\ }\hlstd{VDp}\hlopt{.}\hlstd{}\hlkwd{adjoint}\hlstd{}\hlopt{()\ {*}\ }\hlstd{X}\hlopt{.}\hlstd{}\hlkwd{adjoint}\hlstd{}\hlopt{()\ {*}\ }\hlstd{y}\hlopt{);}\hspace*{\fill}\\
    \hlstd{}\hlkwb{const\ }\hlstd{VectorXd}\hlstd{\ \ \ \ \ \ \ \ \ \ \ \ \ }\hlstd{}\hlkwd{se}\hlstd{}\hlopt{(}\hlstd{s\ }\hlopt{{*}\ }\hlstd{VDp}\hlopt{.}\hlstd{}\hlkwd{rowwise}\hlstd{}\hlopt{().}\hlstd{}\hlkwd{norm}\hlstd{}\hlopt{());}\hlstd{}\hspace*{\fill}
    \normalfont
    \normalsize
  %\end{quote}
  \caption{\textbf{SymmEigLSCpp}: Least squares using the eigendecomposition}
  \label{SymmEigLS}
\end{figure}
% using Eigen::SelfAdjointEigenSolver;

% const SelfAdjointEigenSolver<MatrixXd>
%                           VLV(MatrixXd(p, p).setZero()
%                                             .selfadjointView<Lower>
%                                             .rankUpdate(X.adjoint()));
% const ArrayXd               D(eig.eigenvalues());
% const int                   r((D > D[p - 1] * threshold()).count());
% const MatrixXd            VDp(VLV.eigenvectors() * Dplus(D.sqrt(),r,true));
% const VectorXd        betahat(VDp * VDp.adjoint() * X.adjoint() * y);
% const VectorXd             se(s * VDp.rowwise().norm());

\subsection{Least squares using the column-pivoted QR decomposition}
\label{sec:colPivQR}

The column-pivoted QR decomposition provides results similar to those
from \proglang{R} in both the full-rank and the rank-deficient cases.
The decomposition is of the form
\begin{displaymath}
  \bm X\bm P=\bm Q\bm R=\bm Q_1\bm R_1
\end{displaymath}
where, as before, $\bm Q$ is $n\times n$ and orthogonal and $\bm R$ is
$n\times p$ and upper triangular.  The $p\times p$ matrix $\bm P$ is a
permutation matrix.  That is, its columns are a permutation of the
columns of $\bm I_p$.  It serves to reorder the columns of $\bm X$ so
that the diagonal elements of $\bm R$ are non-increasing in magnitude.

An instance of the class \code{Eigen::ColPivHouseholderQR} has a
\code{rank()} method returning the computational rank of the matrix.
When $\bm X$ is of full rank one can use essentially the same code as
in the unpivoted decomposition except that one must reorder the
standard errors.  When $\bm X$ is rank-deficient, the
coefficients and standard errors are evaluated for the leading $r$ columns of $\bm
X\bm P$ only.

In the rank-deficient case the straightforward calculation of the
fitted values, as $\bm X\widehat{\bm\beta}$, cannot be used.  One
could do some complicated rearrangement of the columns of X and the
coefficient estimates but it is conceptually (and computationally)
easier to employ the relationship
\begin{displaymath}
  \widehat{\bm y} = \bm Q_1\bm Q_1^\prime\bm y=\bm Q
  \begin{bmatrix}
    \bm I_r & \bm 0\\
    \bm 0   & \bm 0
  \end{bmatrix}
  \bm Q^\prime\bm y
\end{displaymath}
The vector $\bm Q^\prime\bm y$ is called the ``effects'' vector in \proglang{R}.
\begin{figure}[htb]
  %\begin{quote}
    \noindent
    \ttfamily
    \hlstd{}\hlkwa{using\ }\hlstd{Eigen}\hlopt{::}\hlstd{ColPivHouseholderQR}\hlopt{;}\hspace*{\fill}\\
    \hlstd{}\hlkwc{typedef\ }\hlstd{ColPivHouseholderQR}\hlopt{$<$}\hlstd{MatrixXd}\hlopt{$>$::}\hlstd{PermutationType}\hlstd{\ \ }\hlstd{Permutation}\hlopt{;}\hspace*{\fill}\\
    \hlstd{}\hspace*{\fill}\\
    \hlkwb{const\ }\hlstd{ColPivHouseholderQR}\hlopt{$<$}\hlstd{MatrixXd}\hlopt{$>$\ }\hlstd{}\hlkwd{PQR}\hlstd{}\hlopt{(}\hlstd{X}\hlopt{);}\hspace*{\fill}\\
    \hlstd{}\hlkwb{const\ }\hlstd{Permutation}\hlstd{\ \ \ \ \ \ \ \ \ \ \ \ \ \ \ \ \ \ \ }\hlstd{}\hlkwd{Pmat}\hlstd{}\hlopt{(}\hlstd{PQR}\hlopt{.}\hlstd{}\hlkwd{colsPermutation}\hlstd{}\hlopt{());}\hspace*{\fill}\\
    \hlstd{}\hlkwb{const\ int}\hlstd{\ \ \ \ \ \ \ \ \ \ \ \ \ \ \ \ \ \ \ \ \ \ \ \ \ \ \ \ \ \ }\hlkwb{}\hlstd{}\hlkwd{r}\hlstd{}\hlopt{(}\hlstd{PQR}\hlopt{.}\hlstd{}\hlkwd{rank}\hlstd{}\hlopt{());}\hspace*{\fill}\\
    \hlstd{VectorXd}\hlstd{\ \ \ \ \ \ \ \ \ \ \ \ \ \ \ \ \ \ \ \ \ \ \ \ \ \ \ \ \ \ \ }\hlstd{betahat}\hlopt{,\ }\hlstd{fitted}\hlopt{,\ }\hlstd{se}\hlopt{;}\hspace*{\fill}\\
    \hlstd{}\hlkwa{if\ }\hlstd{}\hlopt{(}\hlstd{r\ }\hlopt{==\ }\hlstd{X}\hlopt{.}\hlstd{}\hlkwd{cols}\hlstd{}\hlopt{())\ \{\ }\hlstd{}\hlslc{//\ full\ rank\ case}\hspace*{\fill}\\
    \hlstd{}\hlstd{\ \ \ \ }\hlstd{betahat}\hlstd{\ \ }\hlstd{}\hlopt{=\ }\hlstd{PQR}\hlopt{.}\hlstd{}\hlkwd{solve}\hlstd{}\hlopt{(}\hlstd{y}\hlopt{);}\hspace*{\fill}\\
    \hlstd{}\hlstd{\ \ \ \ }\hlstd{fitted}\hlstd{\ \ \ }\hlstd{}\hlopt{=\ }\hlstd{X\ }\hlopt{{*}\ }\hlstd{betahat}\hlopt{;}\hspace*{\fill}\\
    \hlstd{}\hlstd{\ \ \ \ }\hlstd{se}\hlstd{\ \ \ \ \ \ \ }\hlstd{}\hlopt{=\ }\hlstd{Pmat\ }\hlopt{{*}\ }\hlstd{PQR}\hlopt{.}\hlstd{}\hlkwd{matrixQR}\hlstd{}\hlopt{().}\hlstd{}\hlkwd{topRows}\hlstd{}\hlopt{(}\hlstd{p}\hlopt{).}\hlstd{triangularView}\hlopt{$<$}\hlstd{Upper}\hlopt{$>$().}\hspace*{\fill}\\
    \hlstd{}\hlstd{\ \ \ \ \ \ \ \ }\hlstd{}\hlkwd{solve}\hlstd{}\hlopt{(}\hlstd{MatrixXd}\hlopt{::}\hlstd{}\hlkwd{Identity}\hlstd{}\hlopt{(}\hlstd{p}\hlopt{,\ }\hlstd{p}\hlopt{)).}\hlstd{}\hlkwd{rowwise}\hlstd{}\hlopt{().}\hlstd{}\hlkwd{norm}\hlstd{}\hlopt{();}\hspace*{\fill}\\
    \hlstd{}\hlopt{\}\ }\hlstd{}\hlkwa{else\ }\hlstd{}\hlopt{\{}\hspace*{\fill}\\
    \hlstd{}\hlstd{\ \ \ \ }\hlstd{MatrixXd}\hlstd{\ \ \ \ \ \ \ \ \ \ \ \ \ \ \ \ \ \ \ \ \ \ }\hlstd{}\hlkwd{Rinv}\hlstd{}\hlopt{(}\hlstd{PQR}\hlopt{.}\hlstd{}\hlkwd{matrixQR}\hlstd{}\hlopt{().}\hlstd{}\hlkwd{topLeftCorner}\hlstd{}\hlopt{(}\hlstd{r}\hlopt{,\ }\hlstd{r}\hlopt{).}\hspace*{\fill}\\
    \hlstd{}\hlstd{\ \ \ \ \ \ \ \ \ \ \ }\hlstd{triangularView}\hlopt{$<$}\hlstd{Upper}\hlopt{$>$().}\hspace*{\fill}\\
    \hlstd{}\hlstd{\ \ \ \ \ \ \ \ \ \ \ }\hlstd{}\hlkwd{solve}\hlstd{}\hlopt{(}\hlstd{MatrixXd}\hlopt{::}\hlstd{}\hlkwd{Identity}\hlstd{}\hlopt{(}\hlstd{r}\hlopt{,\ }\hlstd{r}\hlopt{)));}\hspace*{\fill}\\
    \hlstd{}\hlstd{\ \ \ \ }\hlstd{VectorXd}\hlstd{\ \ \ \ \ \ \ \ \ \ \ \ \ \ \ \ \ \ \ }\hlstd{}\hlkwd{effects}\hlstd{}\hlopt{(}\hlstd{PQR}\hlopt{.}\hlstd{}\hlkwd{householderQ}\hlstd{}\hlopt{().}\hlstd{}\hlkwd{adjoint}\hlstd{}\hlopt{()\ {*}\ }\hlstd{y}\hlopt{);}\hspace*{\fill}\\
    \hlstd{}\hlstd{\ \ \ \ }\hlstd{betahat}\hlopt{.}\hlstd{}\hlkwd{fill}\hlstd{}\hlopt{(::}\hlstd{NA\textunderscore REAL}\hlopt{);}\hspace*{\fill}\\
    \hlstd{}\hlstd{\ \ \ \ }\hlstd{betahat}\hlopt{.}\hlstd{}\hlkwd{head}\hlstd{}\hlopt{(}\hlstd{r}\hlopt{)}\hlstd{\ \ \ \ \ \ \ \ \ \ \ \ \ \ \ \ \ \ \ \ }\hlopt{=\ }\hlstd{Rinv\ }\hlopt{{*}\ }\hlstd{effects}\hlopt{.}\hlstd{}\hlkwd{head}\hlstd{}\hlopt{(}\hlstd{r}\hlopt{);}\hspace*{\fill}\\
    \hlstd{}\hlstd{\ \ \ \ }\hlstd{betahat}\hlstd{\ \ \ \ \ \ \ \ \ \ \ \ \ \ \ \ \ \ \ \ \ \ \ \ \ \ \ \ }\hlstd{}\hlopt{=\ }\hlstd{Pmat\ }\hlopt{{*}\ }\hlstd{betahat}\hlopt{;}\hspace*{\fill}\\
    \hlstd{}\hlstd{\ \ \ }\hlstd{}\hlslc{//\ create\ fitted\ values\ from\ effects}\hspace*{\fill}\\
    \hlstd{}\hlstd{\ \ \ }\hlstd{}\hlslc{//\ (cannot\ use\ X\ {*}\ betahat\ when\ X\ is\ rank{-}deficient)}\hspace*{\fill}\\
    \hlstd{}\hlstd{\ \ \ \ }\hlstd{effects}\hlopt{.}\hlstd{}\hlkwd{tail}\hlstd{}\hlopt{(}\hlstd{X}\hlopt{.}\hlstd{}\hlkwd{rows}\hlstd{}\hlopt{()\ {-}\ }\hlstd{r}\hlopt{).}\hlstd{}\hlkwd{setZero}\hlstd{}\hlopt{();}\hspace*{\fill}\\
    \hlstd{}\hlstd{\ \ \ \ }\hlstd{fitted}\hlstd{\ \ \ \ \ \ \ \ \ \ \ \ \ \ \ \ \ \ \ \ \ \ \ \ \ \ \ \ \ }\hlstd{}\hlopt{=\ }\hlstd{PQR}\hlopt{.}\hlstd{}\hlkwd{householderQ}\hlstd{}\hlopt{()\ {*}\ }\hlstd{effects}\hlopt{;}\hspace*{\fill}\\
    \hlstd{}\hlstd{\ \ \ \ }\hlstd{se}\hlopt{.}\hlstd{}\hlkwd{fill}\hlstd{}\hlopt{(::}\hlstd{NA\textunderscore REAL}\hlopt{);}\hspace*{\fill}\\
    \hlstd{}\hlstd{\ \ \ \ }\hlstd{se}\hlopt{.}\hlstd{}\hlkwd{head}\hlstd{}\hlopt{(}\hlstd{r}\hlopt{)}\hlstd{\ \ \ \ \ \ \ \ \ \ \ \ \ \ \ \ \ \ \ \ \ \ \ \ \ }\hlopt{=\ }\hlstd{Rinv}\hlopt{.}\hlstd{}\hlkwd{rowwise}\hlstd{}\hlopt{().}\hlstd{}\hlkwd{norm}\hlstd{}\hlopt{();}\hspace*{\fill}\\
    \hlstd{}\hlstd{\ \ \ \ }\hlstd{se}\hlstd{\ \ \ \ \ \ \ \ \ \ \ \ \ \ \ \ \ \ \ \ \ \ \ \ \ \ \ \ \ \ \ \ \ }\hlstd{}\hlopt{=\ }\hlstd{Pmat\ }\hlopt{{*}\ }\hlstd{se}\hlopt{;}\hspace*{\fill}\\
    \hlstd{}\hlopt{\}}\hlstd{}\hspace*{\fill}
    \normalfont
    \normalsize
  %\end{quote}
  \caption{\textbf{ColPivQRLSCpp}: Least squares using the pivoted QR decomposition}
  \label{ColPivQRLS}
\end{figure}
% using Eigen::ColPivHouseholderQR;
% typedef ColPivHouseholderQR<MatrixXd>::PermutationType  Permutation;

% const ColPivHouseholderQR<MatrixXd> PQR(X);
% const Permutation                   Pmat(PQR.colsPermutation());
% const int                              r(PQR.rank());
% VectorXd                               betahat, fitted, se;
% if (r == X.cols()) {	// full rank case
%     betahat  = PQR.solve(y);
%     fitted   = X * betahat;
%     se       = Pmat * PQR.matrixQR().topRows(p).triangularView<Upper>().
% 	       solve(MatrixXd::Identity(p, p)).rowwise().norm();
% } else {
%     MatrixXd                      Rinv(PQR.matrixQR().topLeftCorner(r, r).
% 				       triangularView<Upper>().
% 				       solve(MatrixXd::Identity(r, r)));
%     VectorXd                   effects(PQR.householderQ().adjoint() * y);
%     betahat.fill(::NA_REAL);
%     betahat.head(r)                    = Rinv * effects.head(r);
%     betahat                            = Pmat * betahat;
% 			// create fitted values from effects
% 			// (cannot use X * betahat when X is rank-deficient)
%     effects.tail(X.rows() - r).setZero();
%     fitted                             = PQR.householderQ() * effects;
%     se.fill(::NA_REAL);
%     se.head(r)                         = Rinv.rowwise().norm();
%     se                                 = Pmat * se;
% }

Just to check that the code in Figure~\ref{ColPivQRLS} does indeed provide the desired answer

\begin{verbatim}
> print(summary(fmPQR <- fastLm(y ~ f1 * f2, dd)), signif.stars=FALSE)

Call:
fastLm.formula(formula = y ~ f1 * f2, data = dd)

             Estimate Std. Error t value  Pr(>|t|)
(Intercept)  0.977859   0.058165  16.812 3.413e-09
f1B         12.038068   0.082258 146.346 < 2.2e-16
f1C          3.117222   0.082258  37.896 5.221e-13
f1D          4.068523   0.082258  49.461 2.833e-14
f2b          5.060123   0.082258  61.516 2.593e-15
f2c          5.997592   0.082258  72.912 4.015e-16
f1B:f2b     -3.014763   0.116330 -25.916 3.266e-11
f1C:f2b      7.702999   0.116330  66.217 1.156e-15
f1D:f2b      8.964251   0.116330  77.059 < 2.2e-16
f1B:f2c            NA         NA      NA        NA
f1C:f2c     10.961326   0.116330  94.226 < 2.2e-16
f1D:f2c     12.041081   0.116330 103.508 < 2.2e-16

Residual standard error: 0.2868 on 11 degrees of freedom
Multiple R-squared: 0.9999,	Adjusted R-squared: 0.9999
> all.equal(coef(fm1), coef(fmPQR))
[1] TRUE
> all.equal(unname(fitted(fm1)), fitted(fmPQR))
[1] TRUE
> all.equal(unname(residuals(fm1)), residuals(fmPQR))
[1] TRUE
> 
\end{verbatim}

The rank-revealing SVD method produces the same fitted
values but not the same coefficients.

\begin{verbatim}
> print(summary(fmSVD <- fastLm(y ~ f1 * f2, dd, method=4L)), signif.stars=FALSE)

Call:
fastLm.formula(formula = y ~ f1 * f2, data = dd, method = 4L)

             Estimate Std. Error t value  Pr(>|t|)
(Intercept)  0.977859   0.058165  16.812 3.413e-09
f1B          7.020458   0.038777 181.049 < 2.2e-16
f1C          3.117222   0.082258  37.896 5.221e-13
f1D          4.068523   0.082258  49.461 2.833e-14
f2b          5.060123   0.082258  61.516 2.593e-15
f2c          5.997592   0.082258  72.912 4.015e-16
f1B:f2b      2.002847   0.061311  32.667 2.638e-12
f1C:f2b      7.702999   0.116330  66.217 1.156e-15
f1D:f2b      8.964251   0.116330  77.059 < 2.2e-16
f1B:f2c      5.017610   0.061311  81.838 < 2.2e-16
f1C:f2c     10.961326   0.116330  94.226 < 2.2e-16
f1D:f2c     12.041081   0.116330 103.508 < 2.2e-16

Residual standard error: 0.2868 on 11 degrees of freedom
Multiple R-squared: 0.9999,	Adjusted R-squared: 0.9999
> all.equal(coef(fm1), coef(fmSVD))
[1] "'is.NA' value mismatch: 0 in current 1 in target"
> all.equal(unname(fitted(fm1)), fitted(fmSVD))
[1] TRUE
> all.equal(unname(residuals(fm1)), residuals(fmSVD))
[1] TRUE
> 
\end{verbatim}

The coefficients from the symmetric eigendecomposition method are the same as those from the SVD

\begin{verbatim}
> print(summary(fmVLV <- fastLm(y ~ f1 * f2, dd, method=5L)), signif.stars=FALSE)

Call:
fastLm.formula(formula = y ~ f1 * f2, data = dd, method = 5L)

             Estimate Std. Error t value  Pr(>|t|)
(Intercept)  0.977859   0.058165  16.812 3.413e-09
f1B          7.020458   0.038777 181.049 < 2.2e-16
f1C          3.117222   0.082258  37.896 5.221e-13
f1D          4.068523   0.082258  49.461 2.833e-14
f2b          5.060123   0.082258  61.516 2.593e-15
f2c          5.997592   0.082258  72.912 4.015e-16
f1B:f2b      2.002847   0.061311  32.667 2.638e-12
f1C:f2b      7.702999   0.116330  66.217 1.156e-15
f1D:f2b      8.964251   0.116330  77.059 < 2.2e-16
f1B:f2c      5.017610   0.061311  81.838 < 2.2e-16
f1C:f2c     10.961326   0.116330  94.226 < 2.2e-16
f1D:f2c     12.041081   0.116330 103.508 < 2.2e-16

Residual standard error: 0.2868 on 11 degrees of freedom
Multiple R-squared: 0.9999,	Adjusted R-squared: 0.9999
> all.equal(coef(fmSVD), coef(fmVLV))
[1] TRUE
> all.equal(unname(fitted(fm1)), fitted(fmSVD))
[1] TRUE
> all.equal(unname(residuals(fm1)), residuals(fmSVD))
[1] TRUE
> 
\end{verbatim}

\subsection{Comparative speed}

In the \pkg{RcppEigen} package the \proglang{R} function to fit linear
models using the methods described above is called \code{fastLm}. It follows
an earlier example in the \pkg{Rcpp} package which was carried over to both
\pkg{RcppArmadillo} and \pkg{RcppGSL}. The natural question to ask is, ``Is it indeed fast to use these methods
based on \pkg{Eigen}?''.  To this end, the example provides benchmarking code for these
methods, \proglang{R}'s \code{lm.fit} function and the \code{fastLm}
implementations in the \pkg{RcppArmadillo} \citep{CRAN:RcppArmadillo}
and \pkg{RcppGSL} \citep{CRAN:RcppGSL} packages, if they are
installed.  The benchmark code, which uses the \pkg{rbenchmark}
\citep{CRAN:rbenchmark} package, is in a file named
\code{lmBenchmark.R} in the \code{examples} subdirectory of the
installed \pkg{RcppEigen} package.


It can be run as
\begin{verbatim}
> source(system.file("examples", "lmBenchmark.R", package="RcppEigen"))
\end{verbatim}
Results will vary according to the speed of the processor, the
number of cores and the implementation of the BLAS (Basic Linear
Algebra Subroutines) used.  (\pkg{Eigen} methods do not use the BLAS
but the other methods do.)  

Results obtained on a desktop computer, circa 2010, are shown in
Table~\ref{tab:lmRes}
\begin{table}[tb]
  \caption{\code{lmBenchmark} results on a desktop computer for the
    default size, $100,000\times 40$, full-rank model matrix running
    20 repetitions for each method.  Times (Elapsed, User and Sys) are
    in seconds.  The BLAS in use is a single-threaded version of Atlas
    (Automatically Tuned Linear Algebra System).}
  \label{tab:lmRes}
  \centering
  \begin{tabular}{r r r r r}
    \toprule
    \multicolumn{1}{c}{Method} & \multicolumn{1}{c}{Relative} &
    \multicolumn{1}{c}{Elapsed} & \multicolumn{1}{c}{User} &
    \multicolumn{1}{c}{Sys}\\
    \cmidrule(r){2-5}   % middle rule from cols 2 to 5
     LLt &   1.000000 &   1.227 &     1.228 &    0.000 \\
    LDLt &   1.037490 &   1.273 &     1.272 &    0.000 \\
 SymmEig &   2.895681 &   3.553 &     2.972 &    0.572 \\
      QR &   7.828036 &   9.605 &     8.968 &    0.620 \\
   PivQR &   7.953545 &   9.759 &     9.120 &    0.624 \\
    arma &   8.383048 &  10.286 &    10.277 &    0.000 \\
  lm.fit &  13.782396 &  16.911 &    15.521 &    1.368 \\
     SVD &  54.829666 &  67.276 &    66.321 &    0.912 \\
     GSL & 157.531377 & 193.291 &   192.568 &    0.640 \\
     \bottomrule
  \end{tabular}
\end{table}

These results indicate that methods based on forming and decomposing
$\bm X^\prime\bm X$, (i.e. LDLt, LLt and SymmEig) are considerably
faster than the others.  The SymmEig method, using a rank-revealing
decomposition, would be preferred, although the LDLt method could
probably be modified to be rank-revealing.  Do bear in mind that the
dimensions of the problem will influence the comparative results.
Because there are 100,000 rows in $\bm X$, methods that decompose the
whole $\bm X$ matrix (all the methods except those named above) will
be at a disadvantage.

The pivoted QR method is 1.6 times faster than R's \code{lm.fit} on
this test and provides nearly the same information as \code{lm.fit}.
Methods based on the singular value decomposition (SVD and GSL) are
much slower but, as mentioned above, this is caused in part by $\bm X$
having many more rows than columns.  The GSL method from the GNU
Scientific Library uses an older algorithm for the SVD and is clearly
out of contention.

An SVD method using the Lapack SVD subroutine, \code{dgesv}, may be
faster than the native \pkg{Eigen} implementation of the SVD, which is
not a particularly fast method of evaluating the SVD.

\section{Delayed evaluation}
\label{sec:delayed}

A form of delayed evaluation is used in \pkg{Eigen}.  That is, many
operators and methods do not evaluate the result but instead return an
``expression object'' that is evaluated when needed.  As an example,
even though one writes the $\bm X^\prime\bm X$ evaluation as
\code{.rankUpdate(X.adjoint())} the \code{X.adjoint()} part is not
evaluated immediately.  The \code{rankUpdate} method detects that it
has been passed a matrix that is to be used in its transposed form and
evaluates the update by taking inner products of columns of $\bm X$
instead of rows of $\bm X^\prime$.

Occasionally the method for \code{Rcpp::wrap} will not force an
evaluation when it should.  This is at least what Bill Venables calls
an ``infelicity'' in the code, if not an outright bug.  In the code
for the transpose of an integer matrix shown in Figure~\ref{trans} we
assigned the transpose as a \code{MatrixXi} before returning it with
\code{wrap}.  The assignment forces the evaluation.  If this
step is skipped, as in Figure~\ref{badtrans}, an answer with the correct
shape but incorrect contents is obtained.

\begin{figure}[htb]
  %\begin{quote}
  \noindent
  \ttfamily
  \hlstd{}\hlkwb{const\ }\hlstd{MapMati}\hlstd{\ \ }\hlstd{}\hlkwd{A}\hlstd{}\hlopt{(}\hlstd{as}\hlopt{$<$}\hlstd{MapMati}\hlopt{$>$(}\hlstd{AA}\hlopt{));}\hspace*{\fill}\\
  \hlstd{}\hlkwa{return\ }\hlstd{}\hlkwd{wrap}\hlstd{}\hlopt{(}\hlstd{A}\hlopt{.}\hlstd{}\hlkwd{transpose}\hlstd{}\hlopt{());}\hlstd{}\hspace*{\fill}\\
  \mbox{}
  \normalfont
  \normalsize
  %\end{quote}
  \caption{\textbf{badtransCpp}: Transpose producing incorrect results}
  \label{badtrans}
\end{figure}
\begin{verbatim}
> Ai <- matrix(1:6, ncol=2L)
> ftrans2 <- cxxfunction(signature(AA = "matrix"), badtransCpp, "RcppEigen")
> (At <- ftrans2(Ai))
     [,1] [,2] [,3]
[1,]    1    3    5
[2,]    2    4    6
> all.equal(At, t(Ai))
[1] "Mean relative difference: 0.4285714"
> 
\end{verbatim}

Another recommended practice is to assign objects before wrapping them
for return to \proglang{R}.

\section{Sparse matrices}
\label{sec:sparse}

\pkg{Eigen} provides sparse matrix classes.  An \proglang{R} object of
class \code{dgCMatrix} (from the \pkg{Matrix} package, \citep{CRAN:Matrix}) can be mapped as in Figure~\ref{sparseProd}.

\begin{figure}[htb]
  %\begin{quote}
  \noindent
  \ttfamily
  \hlstd{}\hlkwa{using\ }\hlstd{Eigen}\hlopt{::}\hlstd{MappedSparseMatrix}\hlopt{;}\hspace*{\fill}\\
  \hlstd{}\hlkwa{using\ }\hlstd{Eigen}\hlopt{::}\hlstd{SparseMatrix}\hlopt{;}\hspace*{\fill}\\
  \hlstd{}\hspace*{\fill}\\
  \hlkwb{const\ }\hlstd{MappedSparseMatrix}\hlopt{$<$}\hlstd{}\hlkwb{double}\hlstd{}\hlopt{$>$\ }\hlstd{}\hlkwd{A}\hlstd{}\hlopt{(}\hlstd{as}\hlopt{$<$}\hlstd{MappedSparseMatrix}\hlopt{$<$}\hlstd{}\hlkwb{double}\hlstd{}\hlopt{$>$\ $>$(}\hlstd{AA}\hlopt{));}\hspace*{\fill}\\
  \hlstd{}\hlkwb{const\ }\hlstd{MapVecd}\hlstd{\ \ \ \ \ \ \ \ \ \ \ \ \ \ \ \ \ \ \ \ }\hlstd{}\hlkwd{y}\hlstd{}\hlopt{(}\hlstd{as}\hlopt{$<$}\hlstd{MapVecd}\hlopt{$>$(}\hlstd{yy}\hlopt{));}\hspace*{\fill}\\
  \hlstd{}\hlkwb{const\ }\hlstd{SparseMatrix}\hlopt{$<$}\hlstd{}\hlkwb{double}\hlstd{}\hlopt{$>$}\hlstd{\ \ \ \ \ \ }\hlopt{}\hlstd{}\hlkwd{At}\hlstd{}\hlopt{(}\hlstd{A}\hlopt{.}\hlstd{}\hlkwd{adjoint}\hlstd{}\hlopt{());}\hspace*{\fill}\\
  \hlstd{}\hlkwa{return\ }\hlstd{List}\hlopt{::}\hlstd{}\hlkwd{create}\hlstd{}\hlopt{(}\hlstd{\textunderscore }\hlopt{{[}}\hlstd{}\hlstr{"At"}\hlstd{}\hlopt{{]}}\hlstd{\ \ }\hlopt{=\ }\hlstd{At}\hlopt{,}\hspace*{\fill}\\
  \hlstd{}\hlstd{\ \ \ \ \ \ \ \ \ \ \ \ \ \ \ \ \ \ \ \ }\hlstd{\textunderscore }\hlopt{{[}}\hlstd{}\hlstr{"Aty"}\hlstd{}\hlopt{{]}\ =\ }\hlstd{At\ }\hlopt{{*}\ }\hlstd{y}\hlopt{);}\hlstd{}\hspace*{\fill}\\
  \mbox{}
  \normalfont
  \normalsize
  \caption{sparseProdCpp: Transpose and product with sparse matrices}
  \label{sparseProd}
  %\end{quote}
\end{figure}

\begin{verbatim}
> sparse1 <- cxxfunction(signature(AA = "dgCMatrix", yy = "numeric"),
+                        sparseProdCpp, "RcppEigen", incl)
> data(KNex, package="Matrix")
> rr <- sparse1(KNex$mm, KNex$y)
> stopifnot(all.equal(rr$At, t(KNex$mm)),
+           all.equal(rr$Aty, as.vector(crossprod(KNex$mm, KNex$y))))
> 
\end{verbatim}

Sparse Cholesky decompositions are provided by the
\code{SimplicialLLT} and \code{SimplicialLDLT} classes in the
\pkg{RcppEigen} package for \proglang{R}.  These templated classes are
subclasses of the \code{SimplicialCholesky} templated class in the
\code{unsupported} section of the \pkg{Eigen} library.  A sample usage
is shown in Figure~\ref{fig:spLS}.  The \pkg{RcppEigen} package also
provides the \code{CholmodDecomposition} class which uses the
\pkg{CHOLMOD} sparse matrix decompositions provided in the compiled
code for the \pkg{Matrix} package.  Its use is also shown in
Figure~\ref{fig:spLS}.

\begin{figure}[htb]
  %\begin{quote}
    \noindent
    \ttfamily
    \hlstd{}\hlkwc{typedef\ }\hlstd{Eigen}\hlopt{::}\hlstd{MappedSparseMatrix}\hlopt{$<$}\hlstd{}\hlkwb{double}\hlstd{}\hlopt{$>$}\hlstd{\ \ }\hlopt{}\hlstd{MSpMat}\hlopt{;}\hspace*{\fill}\\
    \hlstd{}\hlkwc{typedef\ }\hlstd{Eigen}\hlopt{::}\hlstd{SparseMatrix}\hlopt{$<$}\hlstd{}\hlkwb{double}\hlstd{}\hlopt{$>$}\hlstd{\ \ \ \ \ \ \ \ \ }\hlopt{}\hlstd{SpMat}\hlopt{;}\hspace*{\fill}\\
    \hlstd{}\hlkwc{typedef\ }\hlstd{Eigen}\hlopt{::}\hlstd{SimplicialLDLt}\hlopt{$<$}\hlstd{SpMat}\hlopt{$>$}\hlstd{\ \ \ \ \ \ \ }\hlopt{}\hlstd{SpChol}\hlopt{;}\hspace*{\fill}\\
    \hlstd{}\hlkwc{typedef\ }\hlstd{Eigen}\hlopt{::}\hlstd{CholmodDecomposition}\hlopt{$<$}\hlstd{SpMat}\hlopt{$>$\ }\hlstd{CholMD}\hlopt{;}\hspace*{\fill}\\
    \hlstd{}\hspace*{\fill}\\
    \hlkwb{const\ }\hlstd{SpMat}\hlstd{\ \ \ \ \ \ }\hlstd{}\hlkwd{At}\hlstd{}\hlopt{(}\hlstd{as}\hlopt{$<$}\hlstd{MSpMat}\hlopt{$>$(}\hlstd{AA}\hlopt{).}\hlstd{}\hlkwd{adjoint}\hlstd{}\hlopt{());}\hspace*{\fill}\\
    \hlstd{}\hlkwb{const\ }\hlstd{VectorXd}\hlstd{\ \ }\hlstd{}\hlkwd{Aty}\hlstd{}\hlopt{(}\hlstd{At\ }\hlopt{{*}\ }\hlstd{as}\hlopt{$<$}\hlstd{MapVecd}\hlopt{$>$(}\hlstd{yy}\hlopt{));}\hspace*{\fill}\\
    \hlstd{}\hlkwb{const\ }\hlstd{SpChol}\hlstd{\ \ \ \ \ }\hlstd{}\hlkwd{Ch}\hlstd{}\hlopt{(}\hlstd{At\ }\hlopt{{*}\ }\hlstd{At}\hlopt{.}\hlstd{}\hlkwd{adjoint}\hlstd{}\hlopt{());}\hspace*{\fill}\\
    \hlstd{}\hlkwa{if\ }\hlstd{}\hlopt{(}\hlstd{Ch}\hlopt{.}\hlstd{}\hlkwd{info}\hlstd{}\hlopt{()\ !=\ }\hlstd{Eigen}\hlopt{::}\hlstd{Success}\hlopt{)\ }\hlstd{}\hlkwa{return\ }\hlstd{R\textunderscore NilValue}\hlopt{;}\hspace*{\fill}\\
    \hlstd{}\hlkwb{const\ }\hlstd{CholMD}\hlstd{\ \ \ \ \ \ }\hlstd{}\hlkwd{L}\hlstd{}\hlopt{(}\hlstd{At}\hlopt{);}\hspace*{\fill}\\
    \hlstd{}\hlkwa{if\ }\hlstd{}\hlopt{(}\hlstd{L}\hlopt{.}\hlstd{}\hlkwd{info}\hlstd{}\hlopt{()\ !=\ }\hlstd{Eigen}\hlopt{::}\hlstd{Success}\hlopt{)\ }\hlstd{}\hlkwa{return\ }\hlstd{R\textunderscore NilValue}\hlopt{;}\hspace*{\fill}\\
    \hlstd{}\hlkwa{return\ }\hlstd{List}\hlopt{::}\hlstd{}\hlkwd{create}\hlstd{}\hlopt{(}\hlstd{\textunderscore }\hlopt{{[}}\hlstd{}\hlstr{"L"}\hlstd{}\hlopt{{]}}\hlstd{\ \ \ \ \ \ \ \ }\hlopt{=\ }\hlstd{}\hlkwd{wrap}\hlstd{}\hlopt{(}\hlstd{L}\hlopt{),}\hspace*{\fill}\\
    \hlstd{}\hlstd{\ \ \ \ \ \ \ \ \ \ \ \ \ \ \ \ \ \ \ \ }\hlstd{\textunderscore }\hlopt{{[}}\hlstd{}\hlstr{"betahatS"}\hlstd{}\hlopt{{]}\ =\ }\hlstd{Ch}\hlopt{.}\hlstd{}\hlkwd{solve}\hlstd{}\hlopt{(}\hlstd{Aty}\hlopt{),}\hspace*{\fill}\\
    \hlstd{}\hlstd{\ \ \ \ \ \ \ \ \ \ \ \ \ \ \ \ \ \ \ \ }\hlstd{\textunderscore }\hlopt{{[}}\hlstd{}\hlstr{"betahatC"}\hlstd{}\hlopt{{]}\ =\ }\hlstd{L}\hlopt{.}\hlstd{}\hlkwd{solve}\hlstd{}\hlopt{(}\hlstd{Aty}\hlopt{),}\hspace*{\fill}\\
    \hlstd{}\hlstd{\ \ \ \ \ \ \ \ \ \ \ \ \ \ \ \ \ \ \ \ }\hlstd{\textunderscore }\hlopt{{[}}\hlstd{}\hlstr{"perm"}\hlstd{}\hlopt{{]}}\hlstd{\ \ \ \ \ }\hlopt{=\ }\hlstd{Ch}\hlopt{.}\hlstd{}\hlkwd{permutationP}\hlstd{}\hlopt{().}\hlstd{}\hlkwd{indices}\hlstd{}\hlopt{());}\hlstd{}\hspace*{\fill}\\
    \mbox{}
    \normalfont
    \normalsize
  %\end{quote}
  \caption{sparseLSCpp: Solving a sparse least squares problem}
  \label{fig:spLS}
\end{figure}

\begin{verbatim}
> sparse2 <- cxxfunction(signature(AA = "dgCMatrix", yy = "numeric"),
+                        sparseLSCpp, "RcppEigen")
> str(rr <-  sparse2(KNex$mm, KNex$y))
List of 4
 $ L       :Formal class 'dCHMsimpl' [package "Matrix"] with 10 slots
  .. ..@ x       : num [1:7395] 1 0.0919 -0.1241 -0.0984 1 ...
  .. ..@ p       : int [1:713] 0 4 8 12 16 24 31 39 46 54 ...
  .. ..@ i       : int [1:7395] 0 694 699 708 1 692 698 707 2 692 ...
  .. ..@ nz      : int [1:712] 4 4 4 4 8 7 8 7 8 7 ...
  .. ..@ nxt     : int [1:714] 1 2 3 4 5 6 7 8 9 10 ...
  .. ..@ prv     : int [1:714] 713 0 1 2 3 4 5 6 7 8 ...
  .. ..@ colcount: int [1:712] 4 4 4 4 8 7 8 7 8 7 ...
  .. ..@ perm    : int [1:712] 256 243 242 241 213 693 212 692 125 633 ...
  .. ..@ type    : int [1:4] 2 0 0 1
  .. ..@ Dim     : int [1:2] 712 712
 $ betahatS: num [1:712] 823 340 473 349 188 ...
 $ betahatC: num [1:712] 823 340 473 349 188 ...
 $ perm    : int [1:712] 120 334 118 332 331 489 333 490 340 131 ...
> res <- as.vector(solve(Ch <- Cholesky(crossprod(KNex$mm)),
+                        crossprod(KNex$mm, KNex$y)))
> stopifnot(all.equal(rr$betahatS, res), all.equal(rr$betahatC, res))
> all.equal(rr$L, Ch)   # unsure why different; Eigen's is smaller
[1] "Attributes: < Component 2: Mean relative difference: 0.4166919 >"  
[2] "Attributes: < Component 4: Numeric: lengths (7395, 7451) differ >" 
[3] "Attributes: < Component 6: Mean relative difference: 0.4166919 >"  
[4] "Attributes: < Component 7: Mean relative difference: 0.0155349 >"  
[5] "Attributes: < Component 8: Mean relative difference: 0.6803645 >"  
[6] "Attributes: < Component 11: Numeric: lengths (7395, 7451) differ >"
> all(rr$perm == Ch@perm) # fill-reducing permutations are different
[1] FALSE
> 
\end{verbatim}

The last two comparisons show that the fill-reducing permutation
computed for the \code{Eigen::CholmodDecomposition} object is
better (in the sense of producing a factor with fewer non-zero
elements) than the one calculated by the \code{Matrix::Cholesky}
function.  This is because the \code{Matrix::CholmodDecomposition} calculation is
performed on $\bm X$ using the COLAMD algorithm whereas the
\code{Matrix::Cholesky} function requires $\bm X^\prime\bm X$ and uses
the AMD algorithm.


\section{Summary}

This paper introduced the \pkg{RcppEigen} package which provides high-level
linear algebra computations as an extension to the \proglang{R} system.
\pkg{RcppEigen} is based on the modern \proglang{C++} library \pkg{Eigen}
which combines extended functionality with excellent performance, and
utilizes \pkg{Rcpp} to interface \proglang{R} with \proglang{C++}.
Several illustrations covered common matrix operations and
several approaches to solving a least squares problem---including an extended
discussion of rank-revealing approaches.  A short example provided
an empirical illustration  for the excellent run-time performance of the
\pkg{RcppEigen} package.

\bibliography{Rcpp}

\end{document}

%%% Local Variables: 
%%% mode: latex
%%% TeX-master: t
%%% End: 



